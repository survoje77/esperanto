\chapter{Senestrecaj principoj}
\section{Libereco}
\indent 
\para{La libereco estas biologia neceso, leĝo de la vivo ; la vivo mem. Ĉiu viva estaĵo, eĉ la plej malsupra, celas, kvankam plej pofte nekonscie, al libero. Eĉ en la vegetaĵaro la vivimpeto estas ekvivalento de la liberimpeto. La ĝermo, kiu elvenas el la grajno, puŝas, sublevas, flankenmetas malhelpojn, aù traiĝas kaj kurbiĝas nesupereblaj baroj kaj serĉas liberan vojon. La malgranda insekto limigita en siaj movoj, kaptita, surpaŝita, enfermita, tiras, puŝas, senĉese serĉas liberiĝon, krias, mordas, pikas, baraktas por almiliti sian liberecon. Estas sovaĝbestoj, kiuj, ĉe plej bonaj materialiaj kondiĉoj kaj plej diligentaj zorgoj flanke de la homo, neeviteble mortas, kiam ili estas senigitaj je libero.}
\para{En la historio de la homa socio, la libereco, la strebo al libero kaj la polibera batalo determinas kaj karakterizas la tutan disvolviĝon de la hjomaro. Libereco estas ĉefa faktoro de la progreso. Neniam la homo por ĉiam rezignaciis, kiam la libero estis de li forprenita. La sklaveco, kiom ajn daùra, kiom ajn firmigita kaj sekurigita ĝi estis de la piedpremintoj de la libero, ĉiam finiĝis per liberiĝo. Kiom da fortostreĉo, kiom da sangoverŝoj, kiom da viktimoj kostis rekonkerado de la libereco ! Sed neniu, anticipe mezurinte la kvanton da elverŝota sango, iam haltis pro la viktimoj, ĉar la libero mem estas la plej alta prezo de la vivo, kaj la vivo mem fariĝas por ĝi rimedo.}
\para{Kiel kondiĉo en la sociaj rilatoj libereco ĉiam signifis progreson, floradon de la kulturo kaj prosperon en ĉiuj sferoj de la sciado kaj homagado. La epokoj de libereco estas la plej lumaj en la homhistorio. Ĝis hodiaù mem ĉio kreita en la sfero de la sciencoj kaj artoj dum la antikve grekaj respublikoj, kiam la libero de la civitanoj estis alte aprecata, daùre restas la bazo de ĉiuj sciencaj disciplinoj. La plej grandaj eltrovoj en scienco kaj tekniko ekaperas en la mezepokaj civitoj.}
\para{En la homa socio, same en la pasinteco kiel hodiaù, la plej karakteriza strebo, de ĉiuj homoj el ĉiuj klasoj kaj sociaj tavoloj, estas la aspiro al libero. Eĉ la plej abomenindaj tiranoj, estas kapablaj kompreni la prezon de la libero, kiam ili estas je ĝi senigitaj. Eĉ la plej grandaj tiranoj, kiam ili batalis por la libero, kompreneble por libero sia, ŝajnis veraj batalantoj kaj revolucioj, kaj sukcesis trompi la popolojn. Ĉu ne estis la fiera kaj kuraĝa \qui{Napoleono} simbolo de la revolucio kaj de la batalo por libereco, antaù ol li superregis Eùropon ? Ĉu li ne plorĝemis, la sama \qui{Napoleono}, por la libero, por libero sia, kiam li troviĝis sur Santka Heleno ? Ĉu ne tiuj estis antaùe la plej kuraĝaj antaùvenantoj, kaj defendantoj de la libero, kiuj pli malfrue starigis la gilotinon, kiel purgatorion kaj sanktan lokon.  Se ni povus scii la intimajn sentojn de \qui{Musolini} en la minuto, kiam li devis ekpendi kun la kapon suben, certe ni trovus la saman plorĝemon por la libero, por la libero de tiu ĉi, kiu plej malnoble kaj plej senskrupule ĝin piedpremis daùre, multe da jarojn post kiam li vendis ĉion, kion li posedis el socialismo~?}
\para{Jes, la homo -- tirano aù sklavo -- scipovas apreci la liberon, sed liberon \emph{sian}. Li estas pri tio tute konscia,  nur kiam li ĝin perdas. Atingi ĉi tiun superegan saĝecon, identigante sian liberon kun alies libereco, signifus ĉesi esti sklavo aù tirano kaj fariĝi homo libera, t.e. liberecana.}
\para{La strebo al libero karakterizas ne nur la individuon, sed ankaù la homgrupojn unuigitajn laù materialaj aù spiritaj interesoj, ligitajn profesie, religie, sente, ekonomie, teritorie, vivmaniere,, etne, nacie, k.c. La grupoj kaj asocioj luktas por sia libero en la kunloĝejoj ; la kunloĝejo batalas por sendependeco, kio signifas liberon rilate al pli supera administra unuo ; la komunumoj batalas por sia aùtonomio, kio signifas por sia libero rilate al la ŝtato ; la popoloj luktas por sia nacia libero ; klasoj batalas kontraù aliaj klasoj kaj sociaj tavoloj por sia libero ; inviduoj kaj civitanoj, organizoj kaj partioj batals por politika libero ; la homoj de scienco kaj arto serĉas sian liberon por krei kaj esplori ; lernantoj kaj studentoj  depostulas liberon ; modernaj pedagogioj levas la infanliberon al la plej grava principo de instruado kaj edukado -- libero, libero, ĉiuj celas al libero, almenaù por si ; oni parolas por libero, levas la liberon al celo de sia agado, al tasko de sia vivo ! Ĉu estas io pli universala ol strebo al libero en la tuta kosmo ?}
\para{Sekve, estas klare, ke la libero estas iu principo, kiu staras, skribita sur multaj standardoj, kiu leviĝas kiel programcelo de multaj sociaj movadoj. Ĝi estas tiom universala principo, ke sen ĝia aplikado certaj homagoj fariĝas neeblaj, kaj identiĝas kun la libero mem. Tiel estas la kazo de la moderna edukado ; tiel estas la kazo de l'arto. Laù la koncepto de la nuntempaj modernaj pedagogoj kaj artistoj, ne estas edukado kaj arto sen libero, ekster la libero. Tamen, ĉiuj sociaj movadoj, kiuj enskribis en sian programon kaj sur sian standardon la liberprincipojn, ne fakte estas veraj adeptoj de la libereco en la pratikado de la vivo. Tiuj ne povas esti veraj partianoj de la liberprincipo, kiuj samtempe akceptas ankaù la aùtoritatoprincipon, eĉ sub modera formo. Aùtoritato kaj libero estas principoj neakordigeblaj, nerepacigeblaj. Ĉia aùtoritato laùdifine, esence kaj en sia manifestiĝo, estas neado, limigo de la libero. Sekve, la konflikto ĉe la plej bona kaj sincera deziro estas neevitebla.}
\para{Tiurilate la diferenco inter aùtoritatuloj kaj anarkiistoj estas kvalita, fundamenta, profunda. Sole nur anarkiismo estas la plej esprimo de la libero-principo ; nur ĝin oni povas konsideri, kiel filozofion de libereco. Sole nur la anarkiistoj estas veraj, logikaj kaj plenaj adeptoj de la libero ; nur ili estas la veraj portantoj de la libero.}
\para{Tial, ne estas hazarde, se en la latinaj lingvoj, la nomo \emph{anarkiisto} identiĝis kun la termino \emph{liberecano}, defendanto aù partiano de la libero. La vorton \emph{libertaire} en la franca, kaj \emph{libertario} en la hispana kaj itala -- kiujn bulgare oni tradukas ordinare \emph{senestracano} -- en ĉiuj latinlingvaj enciklopodioj kaj vortaroj oni donas kiel sinonimon de \emph{anarkiisto}.}
\para{For de ni la intencon aserti, ke inter la adeptoj de aùtoritata principo ne ekzistas sinceraj partianoj de la libero. La diferenco inter aùtoritatuloj kaj anarkiistoj pri tiu punkto ne kuŝas en la sincereco de la intencoj kaj deziroj. Ĝi estas multe pli profunda, ligita al la intenco mem de la faktoj, dum, por la adeptoj de la aùtoritata principo, la libero en la plej bona okazo, kiam ili estas sinceraj defendantoj, estas nur celo, plej ofte fora celo. Por la anarkiistoj, fundamenta principo de ilia doktrino, la libero estas en la sama grado kaj  celo kaj rimedo, tasko kaj vojo, regulo same en internaj reciprokaj organizaj rilatoj, kiel ĝenerale en ĉiutaga vivo en la rilatoj al la aliaj, al la socio.}
\para{La historia sperto kaj la ĉiutaga praktiko pruvis, ke iluzie estas kredi, ke per limigo aù piedpremo de la libero, oni povas realigi la liberon. Nur ties plena uzado en la vivo kaj en la ĉiutagaj rilatoj estas garantio, ke oni venos ĝis plena realiĝo de la socia libereco. Ĉiu, eĉ provizora, transira limigo, prokrasto de la praktikado de la libero, ĉiu uzado de aùtoritato kiu substituiĝas al la libero pro kia ajn motivo, neflekseble kondukas al pluigado de la liberlimigo, al la firmigo de la aùtoritato, kio tradukiĝas per la subpremado kaj finatingas la tiranecon.}
\para{La libero ne agnokas, ne toleras limigon. Nur la neprecizeco kaj malriĉeco de la homlingvo trudas al ni paroli pri liberlimo en certa senco. La libero ne estas la nura principo en la reciprokaj rilatoj. Ĝi ne povas esti rigardata isole kaj el inter la aliaj principoj, kiuj kondiĉas tiujn rilatojn. Konsekvence, se oni parolas pri la limoj de la libereco, tio estas por esprimi tiun harmonion en la aplikado de ĉiuj principoj kondiĉantaj la ensocian vivon.}
