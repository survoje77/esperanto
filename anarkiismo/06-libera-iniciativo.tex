\section*{Libera iniciativo}
\addcontentsline{toc}{section}{Libera iniciativo}
\indent 

\para{La plej por la popolamasoj pereiga konsekvenco de la potencuzado konsistas en tio, ke ĝi faras ilin pasivaj, indiferentaj, kaj mortigas ĉe ili la iniciatemon. La aŭtoritato kreas unuflanke ŝafgvidantojn, aliflanke gregan amason. Siavice, la gregeca, pasiveca, apatieca psikologio kreas favoran medion -- necesan kondiĉon -- por la disvolviĝo, diskresko kaj multobligado de la aŭtoritat–bacilo.}

\para{Sen grego ne estiĝas potenco. \qui{La Boesi} antaŭ kelkaj jarcentoj trovis la fonton de la aŭtoritato en ``libervola servuto''.}

\para{Konsekvence, la manifestiĝo de aktiveco, iniciativo, aktiva sinteno fronte al la homoj kaj la institucioj estas ekneado de la ordonpovo. Iniciativo estas la plej efika kontraŭveneno en la batalo kontraŭ tiu terura epidemio nomita aŭtoritato. La \emph{Sankta Skribo} komenciĝas per fama penso: ``En la komenco estis la Vorto''. En la dekadenca biblio de \qui{Pŝibiŝevski} al la Vorto substituiĝas la sekso. La biblio de ĉiu aŭtoritatulo komenciĝas per alies submetiĝo, obeemo, senagemo, indiferenteco. Se la anarkiistoj havus sian biblion, ĝi devus komenciĝi tiele : ``En la komenco estis iniciativo, kaj la iniciativo estis faro, kaj la faro estis en la iniciativo, ĉar la iniciativo estas neado de la subiĝo, kaj sen subiĝo, sen blinda kaj pasiva sekvado al la gvidantoj ne estas aŭtoritato. Kaj tiel iniciato signifas kontraŭpasivecon, kontraŭsenagemon, kontraŭaŭtoritaton, t.e. senestrecon''. Jen kiun grandegan, nemezureblan, neanstataŭeblan gravecon havas iniciativo; la libera, spontanea kaj aktiva iniciato estas por la anarkiistoj nepre necesa kondiĉo por realigo de la anar–kiistaj rilatoj, socia principo sen kiu, en la homa socio samkiel en la ĉiutagaj homaj rilatoj ne povas malaperi la aŭtoritato. ``Anarkiistoj estu iniciatemaj kaj aktivaj, por esti efektive anarkiistoj !'' -- jen la unua postulo, kiun anarkiismo metas antaŭ siaj adeptoj. ``Homoj de la laboro, homoj, estu aktivaj, prenu ĉiam en viaj manoj viajn proprajn farojn por esti liberaj !'' -- Tion instruas anarkiismo al tiuj, kiujn ĝi vokas al nova mondo de bonstato, paco kaj libereco.}

\para{Nekontesteble iniciativo, rigardate kiel ununura principo, estas ankoraŭ ne sufiĉa por determini la senestrecan karakteron, la senestrecan enhavon de la agoj ; kiel ankaŭ la strebo al libero, rigardate aparte kaj izole, malkonkorde kun la aliaj principoj, kiuj konsistigas la fortikan bazon de vera homa etiko, ne estas sufiĉa por eviti la sklavecon inter la homoj. Ofte okazas, ke la homoj kun granda iniciatemo estas ĝuste tiuj ĉi, kiuj fariĝas gvidantoj, ordonantoj, aŭtoritatuloj, tiranoj kaj diktatoroj. Bremso kontraŭ tiu pasio de gvidanteco, regpovo kaj aŭtoritato povas esti nur iniciato de aliaj homoj, de ĉiuj homoj. Sed se la manifestiĝo de iniciativo ne estas sufiĉa por ke ekzistu senestreco, senestreco ne estas ebla sen manifestiĝo de iniciativo. Homoj sen iniciativo ne estas, ne povas esti kaj neniam fariĝos anarkiistoj, ĉar ili ĉiam bezonas, bezonos la fremdan iniciativon -- iniciativon de la homoj, kiuj kondukas ilin je la nazo. La loko de la homoj sen iniciativo estas inter la sklavoj, kaj la sklavoj ne povas esti akuŝistoj de la libereco.}


\para{Ĉu ekzistas libera iniciativo kiel principo, kaj el kio eligis ĝin la anarkiistoj ?}


\para{Senestreco ne estas dogmo, ne estas skolastiko. Nek elpensita principo aplikota aŭ trudota. Kiel ĉiuj aliaj principoj, tiel ankaŭ la principo de libera iniciativo estas de la anarkiismo konstatebla en la vivo, kaj el ĉi–lasta eligebla por inkludiĝi en sistemo de organizado de la homaj rilatoj. La elmontriĝo de iniciativo, kvankam sub formo de instinkto, estas propra al la tuta animala mondo. Ekzemplo de tia manifestiĝo estas arigo de la bestoj en socioj, aŭ ilia unuopiĝo, aŭ disigo en parojn, grupojn laŭsezone, laŭ ordono de la bezonoj de nutraĵo, de sekureco por plenumi naturajn funkciojn, aŭ simple por havigi ĝojon kaj amuzon. Tiel estas la manifestiĝoj de la cervoj, kiuj kolektiĝas je gregoj por transpasi kune grandajn riverojn, post kio ili denove diskuras. Tiel estas la konduto de l' birdoj, kiuj ĉe proksimiĝo de la aŭtuno, antaŭ eĉ reala malvarmetiĝo, ekgrupiĝas kaj kune flugas ĉiam pli ofte por fine formi flugantarojn pretajn al migrado.}


\para{Ĉe la homo, nature, la manifestiĝo de iniciativo estas ankoraŭ pli ofta fenomeno, kaj ĝi alprenas konscian formon, tiom pli konscian, ke li en sia historia kaj kultura disvolviĝo kiel kolektivo, kaj en sia intelekta perfektiĝo kiel individuo malproksimiĝas de sia besteca origino. Sen manifestiĝo de iniciativo la homa socio estus tute neebla. Kaj elmontriĝo de iniciativo estas nepre necesa kondiĉo por ĉiu progreso en la scienco, en la tekniko, en la socia vivo. Se ne estus iniciativo, ne estus povintaj ekzisti \qui{Kristo}, \qui{Bruno}, \qui{Galileo}, \qui{Pestalocij}, \qui{Ferer}, \qui{Neùtono}, \qui{Pasteùro}, \qui{Einŝtein}, \qui{Garibaldi}, \qui{Bakunin}, \qui{Maĥno}, \qui{Duruti}, bogomilismo, reformacio, franca revolucio, kristanismo, socialismo kaj senestreco. Se ne estus iniciativo, la Historio ne estas vidinta la Unuan Internacion, la kooperativan movadon, la milionojn da sindikataj organizoj, kiuj unuigas la laborulojn, produktantojn, konsumantojn; nek la vaporŝipon, la trajnon, la aviadilon, la radion, la televidon ka\dots la atombombon, bedaŭrinde.}


\para{Nekontesteble, iniciativo, libera iniciativo spontanee manifestita estas fakto, principo, kiel en la naturo, tiel ankaŭ en la homa socio -- en la pasinteco kaj en la nuntempo. La tasko de la anarkiistoj estas certigi al ĝi ankoraŭ pli larĝan kaj pli plenan manifestiĝon en la estonteco, metante ĝin en la fundamenton de la socia dinamiko. Kaj malgraŭ ĉio la manifestiĝo de iniciativo neniam estis universala, nek sufiĉe intensa, ĉaralie nek \qui{Aleksandro} la Makedona, nek \qui{Ĝingis-Ĥano}, nek \qui{Kaligulo}, \qui{Tamerlano}, nek la \qui{Cezar}oj, nek \qui{Petro} la Granda, nek \qui{Napoleono}, nek \qui{Hitler}, nek \qui{Musolini}, nek \qui{Stalin} estus eblaj ; ĉar alie la ŝtato, kapitalismo, kolonia kaj socia sklaveco antaŭlonge estus malaperintaj.}


\para{La libera iniciativo, fonto de kreado, kondiĉo por ĉiu–flanka disvolviĝo, estas baza principo en la organizaj rilatoj de la anarkiistoj. Ĝia plena manifestiĝo estas ebla nur en foresto de regpovo. Konsekvence, la vera libera iniciativo, kiel nelimigita kaj spontanea manifestiĝo de la homa volo, en individua kaj kolektiva senco, havas nenion komunan kun la tiel nomata, tro laŭdata de la kapitalistoj, ``privata iniciativo'', post kiu kamufliĝas nur la ekspluatantaj apetitoj, kiuj volas eternigi la monopolon super la produktrimedoj por plenmano da privilegiuloj. Por la granda plimulto da homoj plene senigitaj aŭ limigitaj en la rajto disponi la produktrimedojn, senigitaj je la plena posedo de siaj laborfruktoj -- senigo kaj limigo kiuj certigas ĉian ekspluatadon de la homo fare de l' homo, kaj ĉian aŭtoritaton, ĉian dominadon super la homo -- por la laboruloj, por la popolamasoj, tiu ĉi ``privata iniciativo'' signifas nenion alian, krom la necesa kondiĉo destinita limigi, malakrigi kaj mortigi la liberan hominiciativon.}
