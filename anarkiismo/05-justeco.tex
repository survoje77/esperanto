\section*{Justeco}
\addcontentsline{toc}{section}{Justeco}
\indent 
\para{``\emph{La justeco}'' -- diras \qui{Prudono}, fondinto de la anarkiisma doktrino -- ``\emph{estas mezuro de ĉiuj homaj agoj}''. Kaj fakte por ĉiuj anarkiistaj teoriistoj, la justeco estas la deirpunkto, la baza ŝtono de la socia konstruaĵo. La titolo mem de la libro\footnote{ Esploroj pri la justeco en politiko kaj ĝia influo sur la virto kaj universala feliĉo}, en kiu \qui{Godwin} unuafoje elmetas senestrecajn konceptojn, pri tio atestas : ``\emph{La justeco}, skribas li, \emph{enkalkulas en si ĉiujn moralajn devigojn... La universala justeco kaj reciproka utilo pli forte unuigas la homojn ol kia ajn subskribita kaj sigelita pergameno.}''}

\para{Ĉe \qui{Prudono} ne estas verko, en kiu justeco ne prezentas ĉeftemon de lia penso, kaj speciale tri grandvolumoj estas dediĉitaj al Justeco en la revolucio kaj en la eklezio. \qui{Bakunin} siavice konsideras la homan justecon "ununure agnoskita de ni''. \qui{Kropotkin}, kiu kronis sian vivon per gravega esploro pri la moralaj instruoj de ĉiuj tempoj -- Etiko -- donis senmortan formulon de la moralo tiel : ``\emph{Sen egaleco ne estas justeco; sen justeco ne ekzistas moralo}''.}

\para{La sento de justeco estas profunde enradikita ne nur en la konceptoj, sed ankaŭ estas parto de la ĉiutaga vivo de la anarkiisto. La belga sociologo kaj filozofo, profesoro \qui{Amon} tion konstatas en sia libro Psikologio de la anarkiisto–socialisto, kiu prezentas fundan kaj dokumento -– plenan esploron de la senestreca psikologio.}

\para{Sed kio estas justeco ? Justeco estas identigo de l’ homo kun la aliaj. Justeco signifas meti sin sur alies lokon, ``enigi en alies haŭton'', meti signon de egaleco inter mi kaj la aliaj, kaj fari al si devigan regulon, ke tion oni faru al aliaj, kion oni deziras por si mem. Tiu regulo kuŝas sur la fundamento de ĉiuj religioj kaj estas konata de ni pere de kristanismo –regulo tiel simpla, tiel kompren–ebla de ĉiuj homoj, sed tiel malfacile aplikebla. ``Ne faru al aliaj tion, kion ci ne deziras, ke ili faru al ci'' aŭ kiel \qui{Kropotkin} ĝin ankoraŭ pli bone klarigis, donante al ĝi, anstataŭ negativa, pozitivan karakteron: ``Faru al aliaj tion, kion ci deziras, ke ili faru al ci ĉe 1a samaj cirkonstancoj''. Ĉu estas pli perfekta regulo de konduto, pli perfekta kaj pli justa, se ni povas tiel diri, ol tia justeco?}

\para{Se la homoj sin gvidus en ĉiuj siaj agoj per tiu regulo, ĉiuj malbonoj, ĉiuj malfeliĉoj, ĉiuj suferoj malaperus el la tero. La aŭtoritato, la leĝoj, la juĝejo, la malliberejo, fine la ŝtato ruiniĝus; la maljustaĵoj ĉesus, la ekspluatado, subpremado, milito, mortigo, venĝo, malamo transformiĝus en nur malagrablan rememoron. Bono kaj malbono, belo kaj malbelo, krueleco kaj humaneco, malamo kaj amo kaj multnombraj aliaj komune uzataj nocioj akirus realan valoron. Homo iĝus efektive saĝa. Mezuri per la sama mezuro kiel por si mem, tiel ankaŭ por la aliaj –tio signifus ne nur neniam fari, sed kaj eĉ neniam pensi kaj deziri al iu malbonon. Identigi la aliajn kun si mem, signifas starigi unuecon en la mondo, harmonion sur la tero; la limojn inter ``mi'' kaj ``ci'' malaperigi, estigi eternan komunikadon de homo al homo esprimatan per bonaj pensoj, bondeziroj, bonfarado.}

\para{Ĉu estas ebla, ĉu estas realigebla tio, aŭ ĉu nur sonĝo de pro revemo malsanaj cerboj, ĉu fantazioj, vizioj de senekvilibraj ermitoj kaj asketoj? Ĉu ekzistas, almenaŭ enĝerme, simila sento ĉe la homo, kaj se ``jes'', el kio ĝi originas, de kie ĝi venis kaj kiel enradikiĝis en la homa animo?}

\para{Justeco estas morala nocio, morala principo. Kaj moraleco, kiel konate, akompanas la homon dum lia tuta evoluo, de la simio ĝis \qui{Darvino}, de \qui{Adamo} ĝis \qui{Kropotkin}. Ne estis, ne estas kaj ne povas esti socio sen moralo. Ja ekzistis socioj kun \qui{Tamerlan}oj, \qui{Makiavel}oj, \qui{Cezar}oj, \qui{Hitler}oj, \qui{Musolini}oj, \qui{Stalin}oj, ankaŭ hodiaŭ estas iliaj heredantoj, kiuj volas piedpremi kaj neniigi ĉian moralon, sufoki ĉian senton de justeco, sed ili ĉiam finiĝis kaj finiĝos per fiasko. En la homarhistorio, ju pli ni estis malproksimaj de la ``jarcento de racio'', des pli la morala sento, la moraleco kunfandiĝis kun la religio, ĉar la religio en sia ekĝermo estis nenio krom primitiva filozofio, provo interpreti la kosmon kun ties kompleksaj fenomenoj kaj starigi kondutregulojn por defendo kaj memkonservo de la homa speco vivanta en socio. Ĉu tamen sekvas el tio, ke la fonto de moralaj ideoj kaj sentoj, pli speciale la fonto de la justeco, estas la kredo je Dio? Ĉu tio signifas, ke la moralo havas supernaturan, diecan originon? La legendo pri kreinto kaj kreitaro estas tute senbaza, ĉar se oni akceptas la ekziston de kreinto kaj kreitaro, la vera kreinto estas la homo kaj Dio ties kreitaĵo.}

\para{La scienco pri moralo, starigita jam laŭ la plej sendisputa maniero, asertas ke tio, kion oni akceptas nomi moraleco, moralo havas tri gradojn de manifestiĝo: la unua, la plej simpla, la plej natura, la plej universala, propra al la bestoj vivantaj en socio, estas interhelpo –- la socia instinkto ; la dua grado propra al la homo, estas frukto de la intelekta sintezo, kondiĉita de la socia instinkto, kaj rezulto de observado kaj eksperimento –- tio estas la justeco ĝuste dirita ; la tria kaj plej alta grado estas memofero, en kiu ekfloras la riĉeco de la spirito, la riĉeco de viveca energio.}

\para{Justeco, kiu ĉi tie estas nia studobjekto, havas kiel senperan fonton la senton de simpatio, aperantan eĉ ĉe la bestoj. Simpatio estas la unua paŝo al interproksimiĝo, la unua flamo, kiu degeliĝas la barilon inter ``Mi'' kaj ``Ci'', la instigo al identigo de la alia kun ``mi''. Tio estas facile konstatebla de ĉiu el ni en la fluanta vivo. Sendube, ni pli inklinas esti justaj al tiuj, kiuj inspiras al ni certan simpation ol al tiuj, kiuj estas al ni fremdaj, indiferentaj kaj malajmikaj. En la ĉiutageco, per la multfoja ripetado kaj ekzercado de la sentoj de simpatio, nature kaj senĝene oni venas al transformo en regulon de tio, kio parte manifestiĝas nur kiel sento, alidirite al transformo de sento en principon. Tiel naskiĝis kaj stariĝis la justeco, sed tute ne laŭ sugesto ``de l’ ĉielo''.}

\para{Natura laŭ sia origino, justeco –- bazo de ĉia moralo -– neniam ĉesis montriĝi en la daŭro de la tuta homhistorio, neniam malaperis, nek malaperos. Ĝi elmontriĝas hodiaŭ mem, eĉ en rabema socio, kie la dominantaj institucioj ĉion faras por disbati la ligilojn de reciprokeco inter la homoj, kaj por aliformigi ĉi–lastajn en ``lupojn unu por alia''. Sendispute, la komprenpovo pri justeco dum jarmiloj fervore, insiste, sisteme sufokita, estas grave malakrigita, multe pli ol libero, solidareco kaj eĉ ol la sento de egalecoj; malgraŭ ĉio ĝi ne estas neniigita. Ĝi neniam ĉesis esti principo en la interhomaj rilatoj, kaj sur ĝi konstruiĝas kutimoj, popolkutimoj, moroj, kiuj reprezentas eĉ hodiaŭ la neskribitajn leĝojn regantajn multe pli grave la homsocion ol la miloj da skribitaj leĝoj nekonataj ofte eĉ de la specialistoj–juristoj.}

\para{Anarkiismo, kiel filozofio kaj doktrino pri la socia rekonstruo, kiel programo kaj agado, kiel batala kaj socia movado, levas alte la principon de justeco, ĉerpinte ĝin el la jarmila historio kaj praktiko de la sociaj rilatoj, kaj ĝin metas en la fundamenton de la socia rekonstruo, kiun ĝi celas efektivigi per organizita kaj decidiga agado de la sklavigitaj kaj senrajtigitaj homamasoj.}