\documentclass[12pt,a4paper,openany,twosideside]{memoir}
\usepackage{ucs}   
\usepackage[utf8]{inputenc}
\usepackage[french]{babel}
\usepackage{graphicx}
\usepackage[left=2cm,right=2cm,top=2cm,bottom=2cm]{geometry}
\usepackage{makeidx}
\usepackage{fancyhdr}
\usepackage{parallel}
\usepackage{hyperref}
%\pagestyle{fancy}

\newcommand{\nombreindice}{Enhavo}
\newlistof{listofindice}{tce}{\nombreindice}


% definition of the commands used for the French ToC;
% \captce for chapters, \sectce for sections and
% \ssectce for subsections
\newcommand\captce[1]{%
  \addcontentsline{tce}{chapter}{\protect\makebox[1.3em][l]{\thechapter}#1}}
\newcommand\sectce[1]{%
  \addcontentsline{tce}{section}{\protect\makebox[2.8em][l]{\thesection}#1}}
\newcommand\ssectce[1]{%
\addcontentsline{tce}{subsection}{\protect\makebox[3em][l]{\thesubsection}#1}}


\newcommand{\para}[1]
{\par{#1}
\vspace{5mm}
}

%\usepackage{fontspec}
%\setmainfont{DejaVu Serif}

\newcommand{\qui}[1]
{\textsc{#1}\index{#1}}

\newcommand{\texto}[2]{
\begin{Parallel}[v]{0.475\textwidth}{0.475\textwidth}
  \ParallelLText{#1}
  \ParallelRText{#2}
\end{Parallel}}

\newcommand{\montitle}{
%\begin{titlepage}
%\fontfamily{phv}\selectfont
\vspace*{\stretch{1}}
\begin{flushright}\HUGE
{\HUGE\bfseries  Alexander Berkman}
\end{flushright}
\hrule
\begin{flushright}
{\Huge Qu'est-ce que l'anarchisme ?\\
\vspace{3mm}
\emph{Kio estas anarkiismo ?}}
\vspace{2cm}
%\includegraphics[scale=0.6]{krishnamurti.jpg}
\end{flushright}

\vspace*{\stretch{2}}
\begin{center}
Éditions Mon \LaTeX
\end{center}
%\end{titlepage}
}




\makeindex
\begin{document}
\montitle
\newpage
\pagestyle{fancy}
\chapterstyle{ell}
\chapter[Introduction]{Introduction}
\captce{Enkonduko}
\section{Des idées fausses}
\sectce{Falsaj ideoj}
\texto{Je veux te parler de l'anarchisme. Je veux te dire ce qu'est l'anarchisme, parce que je pense qu'il serait bon pour toi que tu le saches. Parce que l'anarchisme est encore trop peu connu et que ce que les gens en savent repose en général sur des rumeurs et le plus souvent des idées fausses.}
{Mi volas paroli al vi pri anarkiismo. Mi volas diri al vi, kio estas anarkiismo, ĉar mi pensas, ke estus bone por vi, ke vi sciu pri ĝi. Ĉar anarkiismo estas ankoraŭ tro malmulte konata kaj tio, kion homoj scias pri ĝi, ĝenerale baziĝas sur onidiroj kaj plej ofte miskomprenoj.}
\texto{Je veux t'en parler, parce que je crois que l'anarchisme est la chose la plus noble et la plus puissante à laquelle l'homme ait jamais pensé, la seule chose qui puisse te procurer la liberté et le bien être et apporter au monde paix et félicité.}
{Mi volas rakonti tion al vi, ĉar mi kredas, ke anarkiismo estas la plej nobla kaj plej potenca afero, pri kiu la homo iam pensis, la sola, kiu povas doni al vi liberecon kaj bonfarton kaj alporti al mondo pacon kaj feliĉon.}
\texto{Je veux t'en parler  dans une langue  si simple et limpide que je ne pourrai pas être mal compris. Les grands mots et les formules grandiloquentes ne servent qu'à embrouiller. Une pensée rigoureuse implique un vocabu-laire clair.}
{Mi volas rakonti tion al vi en lingvo tiel simpla kaj klara, ke mi ne povas esti miskomprenata. Pompaj vortoj kaj bombastaj formuloj nur servas por konfuzi. Rigora pensado postulas klaran vortprovizon.}
\texto{Mais avant de te parler de ce qu'est l'anarchisme, je veux te dire ce qu'il \emph{n'est pas}. Tant d'idées fausses ont été répandues sur l'anarchisme que cela est nécessaire. Même les personnes intelligentes en ont parfois une conception erronée. Certains parlent de l'anarchisme sans rien y connaître. Et d'autres mentent à son propos, parce qu'ils veulent te cacher la vérité sur lui.}{Sed antaŭ ol diri al vi, kio estas anarkiismo, mi volas diri al vi tion kio ĝi \emph{ne estas}. Tiom da miskompreniĝoj disvastiĝis pri anarkiismo ke tio estas necesa. Eĉ inteligentaj homoj jafoje ricevas  malĝustan koncepton pri ĝi. Iuj parolas pri anarkiismo sen scii ion pri ĝi. Kaj aliaj mensogas pri ĝi, ĉar ili volas kaŝi al vi la veron pri ĝi.}
\texto{L'anarchisme a de nombreux ennemis, tous te mentiront à son propos.}{Anarkiismo havas multajn malamikojn, ĉiuj mensogos al vi pri tio.}
%\section{Justeco}
\indent 
\para{``\emph{La justeco}'' -- diras \qui{Prudono}}
\vspace{10mm}
(à suivre -- daùrigota)
\chapter{Qu'attends-tu de la vie ?}
\captce{Kion vi atendas de la vivo ?}
\newpage
\printindex
\newpage
\tableofcontents
\clearpage
\listofindice
\clearpage
\end{document}