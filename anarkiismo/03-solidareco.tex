\section*{Solidareco}
\addcontentsline{toc}{section}{Solidareco}
\indent  \para{Simile al la libero, la solidareco estas ankaù biologia neceso, instinkto kaj leĝo de la vivo. La manifestiĝoj de sociemo, interhelpo, solidareco kaj oferemo de la individuoj por la aliaj, en la grupoj, familio, komunumo, en la specio en besta mondo, estas konstatitaj en ĉiuj tempoj.  Ĉiam ili montris la naturesploristoj kaj vojanĝantoj ; filozofoj jam de antikva Grekujo faris sintezojn bazitajn sur interhelpaj kaj solidarecaj faktoj ; poetoj el ĉiuj tempoj prikantis. Sed kiel unua, \qui{Darvino} en sia libro \emph{Origino de l'homo} prezentis tiujn manifestiĝojn, tiel esence havantajn karakterojn de biologia leĝo. Pli malfrue \qui{Kropotkin}, en defendo de la darvinismo kontraù ties misformantoj, dediĉis sin elĉerpe studi tiun specon de manisfestiĝoj, kaj en sia libro\footnote{\textit{ Interhelpo kiel faktoro de progresiva dispolviĝo} \textsc{P. Kropotkin}} argumentis novan doktrinon biologian, kiu fariĝis fundamento de la tuta anarkiisma doktrino.}
\para{Efektive, la funsdamenta kono de la vivo de ĉiuj specioj de la zoologia ŝtuparo -- speciale de la insektoj (abeloj, formikoj, termitoj, kaj tiel plu) ĝis la plej supera besto : la homo -- ebligas konstati la gigantan gravecon de interhelpo, de socia instinkto, de solidareco.}
\para{En la homara historio, la praktika de interhelpo, solidareco kaj asocio same liveris brilan postsignon, manifestiĝinte plej ofte paralele kun la libero. La epokoj de plej granda persona kaj socia libero estas ĝuste tiuj de plej granda socia solidareco.}
\para{Pro tiom da paralelismo inter libereco kaj solidareco en la homa socio, estas substrinke, ke ili kunmiksiĝas en komuna manifestiĝo kaj prezentiĝas kiel du flankoj de unu sama fenomeno. Libero sen solidareco estas neebla~; solidareco sen libero montriĝas nepensebla en socio, kiu disvolviĝas kaj prosperas en sia ekonomia kaj kultura progreso. La respublikoj de anktiva Grekujo, la epoko de kampara komunumo, kaj pli malfrue la mezepokaj liberaj civitoj kun sia kompleksa kaj interpektita reto de plej diversaj asocioj por plej variaj celoj, sin ofte sternintaj sur la internacia tereno, estas la plej elstaraj ekzemploj de alte manifestita socia solidareco, akompanata de universala progreso.}
\para{Novtempe la kooperativo, kiu unuigas centojn da milionoj da homoj en la mondo kun sur loka, nacia kaj eĉ internacia formoj, kaj ampleksas la plej diversformajn aktivecojn ; la laborista asocioj, kiu same kunigas milionojn da homoj sur la tuta terglobo ; ĉiaj aliaj asocioj de loka, nacia kaj internacia karaktero, por la kontentigo de la plej diversaj bezonoj, aù kreitaj por la plej diversaj taskoj, kiuj ofte unuigas la homojn laù korporacioj aù klasoj, por ke ili konduku batalon unuj kontraù aliaj -- estas ankaù nerefutebla pruvo pri ekzisto de la solidareco, kiu, instinkta ĉe la bestoj, transformiĝas ĉe la homo en volitan kaj konscian agadon.}
\para{La socio kiel tuto neniam ĝis hodiaù sukcesis realigi la solidarecon inter ĉiuj siaj membroj ; tion oni ne povas kontesti. La socio, de kiam ekzistas premado kaj ekspluatado, ĉiam estis dividita en kastojn kaj kontraùecajn grupojn, kategoriojn, klasojn interbatalantajn : regantoj, supremantoj, ekspluatantoj kaj regatoj, subprematoj, ekspluatoj kaj ĝuste tiu divido malhelpas la socian solidarecon, kaj ankaù la socialan, kulturan kaj materialan progreson de la socio.}
\para{La socio -- hodiaù antagonista pli ol iam -- estas dividita en klasojn ; divido malutilas al la homaro, ĉar ĝi ne permesas la realigon de la universala homa feliĉo.}
\para{Sed nek en la historia pasinteco, nek hodiaù, la domaĝa divido sukcesis plene neniigi la instinkton de interhelpo, la senton de solidareco, la spiriton de asocio. Eĉ ankaù kiel batalrimedo de sociaj kategorioj, la solidareco potence manifestiĝas en ĉiuj socioj, inkluzive de tiuj, en kiuj iu partia aù persona diktaturo strebas al neniigo de ĉia solidareco, anstataùante ĝin per kazerna reĝimo. Plie, la socia instinkto estas tiom profunde enradikiĝanta en la homo, ke iafoje ĝi nevole transsaltas la limojn de la divido kaj manifestiĝas inter homoj el diversaj klasoj., kiuj tamen konservis iun komunan karakterizaĵon -- la bildon de homo. Sen idealigi tiujn izolajn kazojn, nek doni al ili la aspekton de principo, ni ne povas -- eĉ ankaù la plej partia kaj fanatika adepto de la klasbatalo -- refuti, ke la ĉiutaga vivo liveris multe da tiaj faktoj de socia interhelpo inter homoj el diversaj sociaj grupoj, eĉ militantaj inter si, faktoj gloritaj kaj prikantitaj de la artistoj. Tiuj manifestiĝoj estas signifoplenaj precipe dum periodoj de komunaj plagoj kaj katastrofoj : tertremoj, inundoj, hajloj, incendioj, militoj, k.t.p.}
\para{Sekve, neniu povas nei ke, kiel libereco, solidareco estas same principo en la sociaj rilatoj, agnoskita de formulo : ``lbereco, frateco, egaleco'', kaj hodiaù sukskribita sur multaj standardoj de partipolitikaj organizoj.}
\para{Solidareco estas ankaù unu el la bazaj principoj de anarkiismo. Konvenas agnoski ke, kvankam la solidareco estas vaste praktikata en la reciprokaj rilatoj de la anarkiistoj, eĉ multe pli vaste ol inter ĉiuj aliaj homoj, fakto agnostika eĉ de iliaj malamikoj, en la propagando de la anarkiismaj ideoj tiu principo estas ĝis certa grado ombrita de la preferoj al libero. Kaj tio estas facile klarigebla. En la socia vivo ofte okazas, ke la reago kontraù larĝe disvastigita fenomeno (t.e. la subpremado) iras ĝis troa emfazo de ties kontraùo (la libero). La libero estas tiom tretata, ke la anarkiistoj, ĝiaj solaj fidelaj kaj konsekvencaj defendantoj, skizas sin en sia propagando ĉefe per sia batalo por libero. Kaj pro certaj negativaj montriĝoj, tio donis motivon al iuj anarkiistoj, eĉ konsideri la liberon preskaù kiel unikan principon de la senestreca filozofio.}
\para{La vero, se ni povas permesi al ni gradigon, estas ke solidareco en la anarkiisma koncepto okupas unuan lokon, kaj ke ĝi karakterizas multe pli tiun filozofion ol la libero, sen ke eblas, kompreneble, kontraùmeti ambaù principojn. Se oni devas senestrecon karakterizi per du vortoj, ili sendube estas : filozofio de la solidareco kaj de la libereco. Tiel estas laù la opinio de la teoriiistoj. Kaj \qui{Godwin}, kaj \qui{Prudono}, kaj \qui{Bakunin}, kaj \qui{Kropotkin} kaj ĉiuj aliaj teoriistoj de la anarkiisma socialismo, difinante sian vidpunkton fronte al la nun superregantaj konceptoj kaj sistemo, prenis kiel deirpunkton la neceson starigi solidarecon por certigi la individuan liberon.}
\para{\qui{Bakunin}, kategorie espriminte tiun penson, multfoje emfazis en siaj libroj, prelegoj kaj leteroj : ``\emph{Solidareco estas la unua leĝo, libero, la dua}''. Kaj \qui{Kropotkin} ne nur multfoje esprimis la saman penson, sed li ankaù nutris ĝin per la plej larĝa argumentado en ĉiuj siaj verkoj, precipe en \emph{Etiko} kaj \emph{Interhelpo}. Laù li, kaj tio estas la koncepto de hodiaùaj anarkiistoj, la garantio por réaligi la estontan socion, en kiu la libero trovos sian plej plenan aspekton, kuŝas en la homa sociemo, kiu transformiĝas en konscion, bazon de nova moralo, sen kiu la socia vivo estas neebla.}