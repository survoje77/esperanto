\section*{Libera kontrakto}
\addcontentsline{toc}{section}{Libera kontrakto}
\indent 

\para{Ĝis hodiaŭ la homsocio konas nur tri principojn por la organizado de reciprokaj rilatoj. La unua -- la aŭtoritata -- apogas sin sur la leĝo kaj sur ĉiuj institucioj de trudo kaj perforto, kiuj devenas de ĝi. La homo-civitano, membro de kia ajn socio, akceptante la principon de aŭtoritato, estas devigata procedi tiel, kiel la leĝoj ordonas al li ; kontraŭokaze, minacas lin la puno : la skurgo de la polico, la verdikto de la juĝisto, la karcero de la mallibereja gardisto. La nocioj pri bono kaj malbono, pri honesteco kaj malhonesteco, pri homeco kaj kontraŭhomeco, akiras sian enhavon laŭ senco kaj ordonoj de la leĝoj. Tiel, la homo vidas sin "liberigita" de la bezono manifesti konscion pri devo, pri moralo, pri justeco, pri socia solidareco; al la konscio substituiĝas ĉe li la puntimo. Kiel stimulo, kiel faktoro de la homagoj sufiĉas timo. Ju pli forta estas la aŭtoritato, des pli granda la timo de la puno, kiun ĝi estas kapabla trudi; ju pli certigita estas ``la socia ordo" despli garantiita estas la ``socia harmonio".}

\para{La dua —individuisma principo— estas la homkonscio de la libera persono. Ekzistas neniaj devigoj, neniaj normoj, neniaj sinligoj de la individuo al la socio. Sufiĉas la plena konscio de la individuo, por ke ĉi–lasta agu tiel, ke nenio povas domaĝi alies interesojn kaj rajtojn. Kiam ĉiuj agas sammaniere, la socia ordo, la plej supera socia harmonio de la libereco, estas sufiĉe certigita. Ni fidas al la homo kaj estas konvinkitaj, ke lia konscio progrese disvolviĝos proporcie al lia liberiĝo el la danĝera influo de la privata proprieto, de la ŝtato, religio, milito kaj de ĉiuj institucioj de la hodiaŭa aŭtoritata kaj kapitalista socio, kiel ankaŭ de la restaĵoj de la sklaveca pasinto, kiuj ne malaperos enmomente en socialisma socio. Sed ni ankaŭ estas konvinkitaj —studinte la malsimplan, daŭre pli komplikiĝantan organizadon de la sociaj rilatoj—, ke eĉ ce la plej progresinta kaj la plej perfekta individua konscio, ĉi–lasta neniam sufiĉos por garantii regulan, plenan kaj por ĉiam integralan socian harmonion.}

\para{Restas la tria principo, la principo de la anarkiistoj–socialistoj —tiu de la libera kontrakto. Konforme al tiu ĉi principo la bazo de la sociaj rilatoj estas reciproka kaj konscie akceptita promesligo. Nur tia libera kaj konscie akceptita promesligo estas kapabla garantii la liberecon, la solidarecon, la egalecon, la justecon kaj la liberan iniciativon; nur ĝi estas kapabla same certigi la elmontriĝon de la konscio, disvolvi tiun konscion kaj kontroli, ĉu ĝi ŝanceliĝas kaj montras antaŭsignojn de dekadenco.}

\para{Tio ĉi estas la fundamento de anarkiismo, kaj ĝi devas esti komprenata ne nur de ĉiuj homoj fremdaj al la senestreco, sed kaj precipe de la anarkiistoj mem. Anarkiisto, kiu ne sindevigas kaj ne plenumas sindevigojn, bezonas aŭtoritaton. Kaj se la konscio havas por anarkiistoj ian realan valoron -- kaj ĝi efektive havas tian, kaj ili tion multfoje pruvis en la praktiko -- ĝi manifestiĝas ĝuste en tio, ke ili donas esceptan rolon al la reciprokaj sindevigoj -- je la kontrakto -- kaj tiamaniere faras el la aŭtoritato superfluan principon, same en siaj personaj kaj organizaj manifestiĝoj, kiel ankaŭ en la socio, kiun ili propravole kaj konscie antaŭe akceptis.}

\para{\qui{Prudon} kiel unua starigis la principon de la kontrakto kaj faris ĝin bazangula ŝtono de la tuta sia doktrino. Kaj en tio kuŝas lia plej granda merito, kiel fondinto de la senestreco.}

\para{Post li \qui{Bakunin} ankoraŭ pli bone atentigis speciale pri la specife senestreca senco de la kontrakto kiel manifestiĝo de libera homa volo konscianta la neceson de socio. Kiel por \qui{Prudon}, tiel ankaŭ por \qui{Bakunin} la kontrakto devas havi devigan karakteron -- devigo en la kadro de la libereco, konsekvence devigo kun provizora karaktero. ``La forto de la kontrakta devigo --asertis \qui{Bakunin} -- estas limigita. La devigo eterna ne povas esti agnoskita de la homa justeco, ĉiuj rajtoj kaj devoj baziĝas sur la libero.''}

\para{\qui{Kropotkin}, akceptante la saman principon kun ĝia kondiĉa devigo kaj ĝia libera karaktero, siavice findisvolvis la ideon pri libera kontrakto, donis al ĝi sciencan pravigon kaj, kvazaŭ por sin distingi de \qui{Prudon} kaj \qui{Bakunin}, nomis la kontrakton ``libera interkonsento''.}

\para{En kio konsistas kaj kiel devas esti komprenita la principo de libera kontrakto aŭ de libera interkonsento ; kiel devas esti komprenita ĝia libera kaj samtempe deviga karaktero ?}

\para{La libera kontrakto, aplikata egale en la ekonomiaj rilatoj, en la politikaj, sociaj kaj kulturaj manifestiĝoj homaj, en limigita aŭ en larĝa dimensio, prezentas juran bazon de libera senestreca socio. Homoj havante intereson komune fari decidojn starigas formojn de interrilatoj pri difinitaj, provizoraj aŭ longdaŭraj taskoj, determinas laŭ libervola konsento siajn rajtojn kaj reciprokajn devojn. Ĉiuj tiuj decidoj devontigas nur tiujn, kiuj ilin akceptos. Sed post akcepto ili fariĝas devigaj tiom longe dum tiuj, kiuj libervole sin devigis, ne deklaras, ke ili rezignas same la rajtojn, kiel la devojn priskribitajn en la libera kontrakto. Do la devigo ne nur ne estas eterna, ĝi eĉ ne etendiĝas sur la tuta anticipe interkonsentita limtempo por tiuj, kiuj rezignus antaŭ la finiĝo de la kontrakto; sed ĝis kiam ili ne ĝin plenlibere rezignas kaj ĉe egalaj kondiĉoj, tiu ĉi kontrakto restas por ili deviga.}

\para{Ĉu ekzistas tiu ĉi principo en la nuntempa socio, ĉu ĝi manifestiĝis en la historio de la homaro, aŭ ĉu ĝi estas frukto de teoria sintezo?}

\para{Doni kompletan kaj definitivan respondon al tiu ĉi demando signifus rakonti la historion de la mezepokaj liberaj civitoj, de ties multnombraj kaj plej diversaj asocioj pli–malpli longdaŭraj por plej diversaj bezonoj; signifus same trarigardi paŝon post paŝo la ekĝermon kaj evoluon de la kooperativa movado, de la laboristaj profesiaj organizoj, de la \emph{Ruĝa Kruco}, de la diversaj humanecaj organizoj kaj aliaj plej diversaj asocioj; trastudi la ekaperon kaj evoluon de multaj internaciaj ekonomiaj kaj aliaj organizoj, de la fervojaj kaj aliaj internaciaj interkomunikigoj ktp, ktp -- laboro, kiun multego da progresemaj historiistoj kaj sociologoj jam efektivigis; speciale \qui{Kropotkin} majstre tion prezentis en siaj libroj kaj ĉefe en Pano kaj Libero, La moderna scienco kaj anarkio, kaj Interhelpo, al kiuj ni direktas niajn legantojn.}

\para{Senkonteste ĉiuj tiuj ekzemploj, kiujn oni povas montri same en la historio kiel ankaŭ en la nuntempo, havas nur karakteron de ekzemploj, nenion pli. Ili ne estas kaj povas neniaokaze servi kiel modeloj, ĉar ili portas la difektojn de la socia medio, en kiu ili naskiĝis. La estonteco kaj nur la estonteco kreos la perfektajn formojn de la libera kontrakto.}