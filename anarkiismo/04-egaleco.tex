\section{Egaleco}
\indent 
\para{Egaleco proklamita, kune kun libero kaj frateco, fare de la granda franca revolucio, estas nekontesteble principo de sociaj rilatoj, kaj ne nur por la anarkiistoj, kiuj ĝin konsideras kiel la fundamentan principon de sia socia doktrino. El la buŝoj de multaj ni hodiaù aùdas prononci senĝene tiujn tri tiel enhavoriĉajn kaj esprimplenajn vortojn~: ``\emph{liberon, fratecon kaj egalecon}'', kiel kanton, kiel mondkanzonon. Ankoraù pli, tiuj tri vortoj ornamas eĉ la enirejon de iuj publikaj konstruaĵoj kiel iu kruela ironio de la historio. Tiuj grandaj principoj de la homprogreso perdis per trouzado sian veran sencon. Sed se la libero kaj la frateco, aù pli precize la solidareco, ne estas tute senigitaj je sia vera enhavo, kaj havas certan kvankam limigitan sencon por tiuj, kiuj arbitre ornamas sin per ili, ne tiel estas pri egaleco. Cetere, al la nocio egaleco neniam estis donita ĝia vera senco, eĉ dum la tempo de la granda franca revolucio. El la famaj grandaj nomoj, nur tiun de \qui{Marat} oni povas ligi kun iu pli preciza kompreno de egaleco. Por ĉiuj aliaj la proklamado kaj serĉado de egaleco neniam iris pli fore ol egaleco antaù la leĝoj. Ĉiuj, kiuj  parolis kaj parolas pri egaleco, ne enmetas en tiun principon alian enhavon, krom leĝebla, jura kaj abstrakta. Nur en la koncepto de la anarkiistoj, en ilia sociala programo, la principo de egaleco trovas plenan, verecan kaj realan esprimon. En tiu trilogio ``\emph{libereco, solidareco kaj egaleco}'', egaleco por anarkiistoj havas unuarangan kaj determinitan karakteron. La libero neniam povas esti tute reala, se mankas reala egaleco, egaleco en la materialaj kondiĉoj de la vivo. Se ni akceptas, ke la ideala, politika demokratio realiĝu, ke efektiviĝu plena egaleco de la civitanoj antaù la leĝoj, ke ideala konstitucio aplikata de same idealaj ``organoj de la ideala'' demokratia ŝtato absolute garantiu la plenan juran egalecon, la efektiva egaleco eĉ en tiu kazo fariĝos iluzio, se tiam daùre ekzistos la materiala neegaleco. La privata proprieto, la kapitalismo privata kaj ŝtata kondiĉo kaj ĉien akompanata de monopolo, privilegioj kaj herarkio kreas materialan neegalecon. Kian realan valoron povas havi por proleto, por malriĉa kamparano, por eta oficisto kaj metiisto, ilia egaleco-rajto, fronte al la kapitalisto, al la granda proprietulo, al la industriisto, al la supera burokrato, se ili ne havas egalan kaj realan  eblecon plene kontentigi siajn materialajn kaj intelektajn bezonojn ? Tiu egaleco povus esti enskribita en la libroj per oraj literoj, povus esti garantiita de anĝeloj kaj ĉefanĝeloj, prikantita sur ĉiuj vojoj kaj krucvojoj, en ĉiuj parlamentoj kaj tribunaloj, ĝi restus iluzia tiom longe dum ne ekzistas ekonomia egaleco.}
\para{Solidareco neniam povas esti natura, senĝena kaj plena, sen egaleco. Kvankam profunde biologia bezono, la solidareco en la socia vivo estas influata de la socia strukturo, determinita de la rilatoj kiuj karakterizas la socion, kaj alprenas formojn determinitajn de tiuj rilatoj. Tiom longe dum la socio daùre estos dividita en klasojn de regantoj kaj regatoj, de privilegiuloj kaj senrajtigitoj, de espluatantoj kaj ekspluatatoj, tiom longe dum la homoj restos dividitaj per kompleksa sistemo hierarkia, kreanta tutan priramidon de sociaj tavoloj kun kontraùecaj interesoj, la socia solidareco neniam povos realiĝi kiel universala principo.}
\para{Neniu pli bone tion esprimis ol \qui{Bakunin}, difinante la gravecon de egaleco, kiel neanstataùebla principo de socialismo : ``\emph{Libero sen socialismo} (t.e. sen egaleco) \emph{signifas privilegion kaj maljustecon}''.}
\para{Kaj tiel, por anarkiismo kaj la anarkiistoj, egaleco aperas ne nur kiel natura rajto, kiun neniu homa leĝo garantias, sed kiu plej ofte estas de la leĝo piedpremita, nekonsiderata, kaj kiu kontraùas al la proprietrajtoj kaj principoj de socia hierarkio sanktigitaj de ĉiuj konstitucioj de la mondo. Ĝi estas la rajto de libera aliro de ĉiuj homoj al ĉiuj naturaj bonaĵoj kaj al libera utiligo de ĉiuj materialaj kaj intelektaj akiraĵoj de la socio, fruktoj de la nemezureblaj penadoj de milionoj da homoj kaj de nenombreblaj generacioj ; sed egaleco  estas ĉefa kondiĉo por certigi la socian kaj individuan liberecon kaj la rilatojn de plena socia solidareco, kil de naskiĝo kaj tenado de la sento de justeco, fundamento de la socia moralo kaj de ĉia homa etiko ĝenertale.}
\para{Jarcentoj da sklaveco kaj ekspluatado, de subpremado kaj maljusteco lasis la plej profundan postsignojn en la homkonscio koncerne la egalecon, kaj reduktis ĝis minimumo ties manifestiĝojn en la homrilatoj. Dum la konscio pri libero kaj sento de solidareco plu manifestiĝas gravmaniere sub unu aù alia formo, la sento pri egaleco inter la homoj  senteble difektiĝis. Sed tamen ni trovas ĝin respegulitan en diversaj religioj -- primitivan filozofion de homoj el nememoreblaj tempoj -- en multaj sektoj  per kiuj reagis la spirito de la primara kaj pura vero de la religia filozofio kaj precipe  en la kristanismo kaj en la kristanal sektoj el diversaj epokoj. Ni retrovas ĝian manifestiĝon en la kamparana komunumo, en la slava famili-komunumo (\emph{zadrouga} slave), en la diversaj frataroj, unuiĝoj kaj asocioj de la mezepokaj liberaj civitoj. Ni trovas restaĵojn de tiu sento de egaleco en iuj popolaj tradicioj kaj popoldiroj. Ne mankas fine manifestiĝoj de konceptoj kaj de sento pri egaleco ankaù en la homrilatoj eĉ en la hodiaiùa socio tiel profunde dividita kaj disŝirita pro konstantaj kontraùecoj. Sen atribui troan gravecon al tiuj lastaj faktoj, oni devas ĝin konfesi, havas izolan karakteron, ni ne povas ne konstati, ke la pliiĝo de la komercaĉa spirito kun la disvolviĝo de l'kapitalismo ne atingis tian gradon, ke ekzemple la homoj estas pesitaj per pesilo antaù envagoniĝi, envaporŝipiĝi, enaviadiliĝi, enautobusiĝi, kaj pag transportajn taskojn konforme al sia pezo. Io pli, ĉiuj vojanĝantoj estas egle traktataj rilate al pakaĵo senpage transportebla ; ankaù tio valoras iagrade por la peraviadilaj vojaĝoj. Egala traktado aplikiĝas same en la lernejoj, kie la klerigo ĝis fiksita aĝo estas absolute senpaga, kaj per diversaj socialaj kriterioj, fariĝas materiale ebla por ĉiuj infanoj, eĉ ankaù por tiuj de la plej malriĉaj gepatroj. La publikaj bibliotekoj estas alia brila manifestiĝo de egaleco en la kadroj de la hodiaùa rabema socio. Ĉiu povas hodiaù libere kaj senpage utiligi la sociajn bibliotekojn je egalaj kondiĉoj. Ĉe iuj grandaj okcidenteùropaj bibliotekoj estas eĉ specialaj oficistoj, kiuj donas eblecon utiligi bibliotekojn el aliaj urboj kaj landoj, kies deziritaj libroj estas transsenditaj al donita loka biblioteko, kie ili restas dum fiksita tempo, postulante neniun pagon por tiu servo. Tiu rajto estas egala same por princo kaj kapitalisto, kiel por la lasta proleto. Fakte, la materialaj vivkondiĉoj estas ja diversaj, do diversaj estas la ebloj pri la libera tempo, kiun oni dediĉas al vizito al tiuj bibliotekoj. Sed rilate tion, troviĝas ankaù novaj formoj de utiligo de biblioteko, per enkonduko de la koresponda utiligo per tiel nomataj transproteblaj bibliotekoj. Egala kaj nelimigita estas en tiuj urboj utiligo de la akvo en la hejmoj. La saman principon oni aplikas ankaù en restoracioj rilate al pano, kiun oni ne pagas aparte, kaj kies kvanto estas nelimigita por la konsumanto. Multe da tiaj faktoj oni povas nombri. En multaj okazoj ili respondas pli al neceso teknika aù komercaĉa, ol al sento de egaleco ; sed malgraù ĉi tio ili estas pruvoj, kvankam vere sensignifaj, pri ekzisto de la sento de egaleco.}
\para{Sed kio estas egaleco ? Ankaù pri tiu demando la diferencoj inter la anarkiistoj-socialistoj kaj la aliaj socialistoj estas profundaj. Por la anarkiistoj egaleco ne estas kvanta, sed morala nocio : ĝi koincidas kun la rajto de vivo. La marksisma socialismo kaj komunismo estas antaù ĉio produkta sistemo, dum la anarkiisma komunismo estas ĉefe distribua, konsuma. Post kiam anarkiismo neas la parazitismon, aldirite la rajton vivi je fremda kosto certigita per la monopolo super la rimedoj de produktado, interŝanĝado  kaj dispartoigado, t.e. post kiam ĝi proklamas la neceson de socialigo de tiuj rimedoj, ĝi lokas sur la unuan rangon ĉies rajton ĝui ĉiujn materialajn kaj intelektajn havaĵojn konforme al siaj bezonoj, kun la sama natura devigo partopreni en la produktado de tiuj havaĵoj laù siaj fortoj kaj kapabloj. La kontentigo de la bezonoj estas tiom pli plena, kiom pli alta estos la produkta kaj teknika nivelo. La utiligo estos plena kaj nelimigita por la produktoj, kiuj estos en sufiĉa kvanto. Koncerne ĉiujn aliajn bezonojn, la socio zorgos pri la havebla kvanto, kaj ilia dispartigo estos farita laù la grado de la bezonoj.}
\para{Por anakiistoj egaleco ne estas aritmetiko, apotekaj pesiloj ; kvankam il sin apogas sur la socia ekonomio, por ili egaleco ankaù ne estas ekonomio, etik-morala principo. La anarkiisma socialismo estas etika principo ; tiu tendenco manifestiĝas eĉ en la hodiaùa socio kaj ĉe homoj, kiuj ofte havas neniun nocion pri socialismo. En la okazoj de manko, de nesufiĉeco, de plagoj, kaj en la socio, kaj en la familio, ĉe distribuo de nesufiĉecaj havaĵoj preferon oni donas ne al tiuj, kiuj havas la plej grandajn meritojn pro sia produktado, sed al infanoj, malsanuloj, maljunuloj, virinoj, malbonfartantoj. Ankaù tie ĉi anarkiismo aperas pli proksima al la homoj, pli natura ol kalkulanta socialismo, kiu gvidis al monstra staĥanovismo, al vetlaborado miksita kun partia falsado kaj la neado de ĉia humaneco levita ĝis alteco de superega principo.}
\para{La vera egaleco ne praviĝas per meritoj, kvalifiko, kapaboj, hereditaj aù akiritaj rajtoj kaj hierarkio, sed per absoluta agnoskado de la neŝanĝebla valoro de ĉia homa estaĵo. La vera egaleco tuŝas ne nur la materialan sed ankaù la moralan sferon de la homaj manifestiĝoj ; ĉia subpremoj, ĉia servutemo, ĉia subtaksado aù supertaksado de la individuoj, signifas neadon de la egaleco. Ĝi estas konscio pri digno, kiu, sen rifuzi la respekton al homa individuo, malestimas ĉian aùtoritaton. Jen pro kio, la vera egaleco estas antiaùtoritata, senestreca ; kie estas aùtoritato, regpovo, tie ne estas kaj ne povas esti egaleco.}

