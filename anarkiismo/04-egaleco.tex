\section{Egaleco}
\indent 
\para{Egaleco proklamita, kune kun libero kaj frateco, fare de la granda franca revolucio, estas nekontesteble principo de sociaj rilatoj, kaj ne nur por la anarkiistoj, kiuj ĝin konsideras kiel la fundamentan principon de sia socia doktrino. El la buŝoj de multaj ni hodiaù aùdas prononci senĝene tiujn tri tiel enhavoriĉajn kaj esprimplenajn vortojn : ``\emph{liberon, fratecon kaj egalecon}'', kiel kanton, kiel mondkanzonon. Ankoraù pli, tiuj tri vortoj ornamas eĉ la enirejon de iuj publikaj konstruaĵoj kiel iu kruela ironio de la historio. Tiuj grandaj principoj de la homprogreso perdis per trouzado sian veran sencon. Sed se la libero kaj la frateco, aù pli precize la solidareco, ne estas tute senigitaj je sia vera enhavo, kaj havas certan kvankam limigitan sencon por tiuj, kiuj arbitre ornamas sin per ili, ne tiel estas pri egaleco. Cetere, al la nocio egaleco neniam estis donita ĝia vera senco, eĉ dum la tempo de la granda franca revolucio. El la famaj grandaj nomoj, nur tiun de \qui{Marat} oni povas ligi kun iu pli preciza kompreno de egaleco.}