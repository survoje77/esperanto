
\section*{Internaciismo}
\addcontentsline{toc}{section}{Internaciismo}
\indent 

\para{Patriotismo estas alia bildo, sub kiu aperas "malbonaùgura historia fantomo", la dua relaborita kaj laika eldono de la religio. Ĝi estas la nova konfesio, la religio de la moderna ŝtato. Dio tie ĉi alprenas pli materiigitan formon, kaj nomiĝas \emph{Patrujo}. La bolŝevika varianto de patriotismo estas iomete modiifita, sed la senco estas konservita. La diferenco kun la klasika formo de la burĝara patriotismo estas, ke la \emph{patrolando} prezentiĝas en du gradoj : unua estas, la \emph{granda patrujo} -- la \emph{patrujo de la prroletaro}, la Moskva Imperio; kaj la dua drado, \emph{malgranda patrujo}, monopolo de ĉiu satelito kiu kredas kaj krucsignas al la kremlaj dioj. Monoteismo transformiĝas, "evoluas" al politeismo.}
\para{Anarkiismo refutas unu kaj la alian kaj ĉiujn religiojn laù la sama logiko, laù la sama neceso ; same kiel ĉia kulto, ĉia idolo, ĉia fetiĉo estas neado de l'homo, de lia libero, de lia moralo, de lia indeco, kaj de liaj necedeblaj rajtoj.}
\para{Kaj patriotismo samkiel kredo je Dio havas naturan originon. Ĝi ĉerpas siajn unuajn nutrajn sukojn el sento propra ankaù al la bestoj, sento de alligeteco al sia modio. Tio estas la natura, primara patriotismo. ``\textit{Alligiteco instinkta, aùtomata, pleno senigita je kritikemo kontraù la kutimoj tradiciaj, heredaj, kolektivaj, kaj malamo same instinkta, aùtomata kontraù ĉia alia vivmaniero ; amo al la samgentanoj kaj al propra apartenaĵo kaj malamo kontraù ĉio fremda}'' -- diras \qui{bakunin}. Sento, kiu sin manifestas duoble : 
\begin{itemize}[itemsep=0mm]
\item sento de kolektiva egoismo, unuflanke
\item kaj de neado, de milito aliflanke
\end{itemize}}
\para{Kiom ajn primitive kaj sovaĝe sin manifestis tiu natura patriotismo, ĝi ne estas tiel fatala, kaj en la historio ne ludis tian pereigan rolon kiel la patriotismo, kiu ligita kun kreo de la nacia ŝtato, transfomiĝas en religion. En la unua kazo ĝi estas reala fakto, dum en la dua ĝi transformiĝas en principon, en doktrinon, en idealon iniciatitan, konscie, sisteme kaj prointerese instruitan kaj disvatigatan kaj kulturatan. Reala fakto de limigita kaj fermita solidareco, respondanta al iu grado de evoluo, ĉe kio oni ne povas atendi nek postuli pli ol tio, kion la bestohomo kapablas doni ; la patriotismo, transformita en principon kaj altigita al rango de nacia religio, fariĝas neado de solidareco mem kaj konstanta minaco por la universala paco, por la homlibereco, por la justeco kaj por la justemo mem inter la homoj, eĉ enlime de malgranda kaj limigita kolektivo. Duobla sento : de amo, de korinklino -- kaj eĉ ekskluziva ; kaj samtempe de malamo ; kiam patriotismo iĝas doktrino instruata de la supro de profesora katedro, la malamo al fremdaĵo iĝas superanto elemento laù sia enhavo kaj sufokas la manifestiĝon de amo kaj korinklino al si mem kaj al la samgentanoj kaj al ĉia propra apartenaĵo. Oni ne forgesu, ke la homo estas besto, kaj ju pli kulturiĝas lia primara besta naturo, despli superiĝas la negativaj instinktoj.}
\para{Patriotismo kaj religio de la malamo al la fremdaĵo estas la plej potenca rimedo por sovaĝbestigo de la homo. Sed patriotismo ne estas nur primara, natura sento de ``kolektiva egoismo''. En sia nuntempa esprimo de nacia religio, tiu natura ĉefeco preskaù perdiĝas ; nur la esploristo de la patriotisma deveno povas ĝin konstati ; ĝi anstataùiĝas de interesoj ekonomiaj kaj politikaj, kaj de fanatikeco. En tiu formo patriotismo estas por la senestreco objekto de kritiko, de atrakoj kaj neado. Ĉar la ekonomiaj kaj politikaj interesoj, kiujn la patriotismo reprezentas, kaj la fanatikeco, kiu al ĝi donas la forton de amasa tokso, estas la necesaj kondiĉoj sen kiuj la nuntempa modernaŝtato nacia ne povus eksisti, sendepende de tio ĉu ĝi estas aù sin nomas demokratia, burĝara aù ``socialisma kaj proletara.''}
\para{Kiuj kaj kiaj estas tiuj interesoj, kia estas kaj kiel sin manifestas tiu fanatikeco ?}
\para{La ekonomiaj interesoj konsistas en ekskluziva monopolo super donita teritorio, kiu baziĝas sur ``historiaj rajtoj'', t.e.sur la ĉefeco de la konkerado, sur la forrabo -- la rajto de la plej forta~-- aù sur la neceseco de ``vivesenca spaco'', alivorte la rajto de la impertinentulo. La ekonomiaj interesoj ankaù konsistas  en ekluziva rajto posedi kaj uzi la subterajn kaj surteraj riĉaĵon de tiu teritorio, kaj en la avida strebo kaj ebleco certigita per akrigita patriotismo, konkeri je unu aù alia preteksto ankaù fremdajn kaj ofte eĉ forajn landojn kun ties naturaj riĉaĵoj.}