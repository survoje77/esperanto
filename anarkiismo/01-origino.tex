\chapter[Origino kaj movado]{Origino de la senestrecaj ideoj\\kaj movado}

\para{Same kiel la ideoj pri aùtoritato, ankaù la ideoj pri senestreco havas naturkaùzan originon. Ili ne estas falintaj de la ĉielo, sed venas de la vivo. Ili ekradikas el la homo, el lia naturo. Ili naskiĝas el materialaj kaùzoj en la plej vasta senco de tiu ĉi nocio. Ili estas frukto de la komplekso de kondiĉo, meze de kiuj la homo naskiĝas, vivas, disvolviĝas, agas aù eltenas efikojn, kreas kaj konsumas havaĵojn, lernas, luktas, revas, ĝuas aù suferas. Sed la homa naturo estas kontraùudira kaj kompleksa. Tial, ne nur la senestrecaj ideoj sed ankaù la aùtoritatal ideoj same fontas el la homo.}

\para{La instinkto de memkonservado kaj la seksa instinkto neprigas reciprokan lukton, kiu aperas en diversaj kaj multaspektaj formoj ; ili estas fonto de aùtoritato. La historio de la homaro estas plena de de tiaj pereigaj manifestiĝoj. Ĝuste tiuj manifestiĝoj  de instinkta aùtoritato estas fontoj de la aùtoritataj ideoj, hodiaù esprimitaj en kompleta filozofio, kiu metas la instituciojn super la homo, la formulojn super la vivo, kreas kulton al la ŝtato, kaj transformas la homon en objekton, aùtomaton, regaton.}
\para{La sento de libereco, unuflanke, kiu ne kontraùas al la instinkto de memkonservado kaj ne nur ne kontraùstaras la lukton sed eĉ ofte ĝin kondiĉas, la socia instinkto, aliflanke, kiu ne kontraùstaras la memkonservadan instinkton, sed ĝin plivastigas kaj donas al la lukto aliajn formojn kaj aliajn sencojn estas esencaj fontoj de la ideoj pri senestreco.}
\para{De tiu vidpunkto la tuta homara historio lumiĝas diference de la historio instruita en la ŝtataj lernejoj. La disvolviĝo de la homo kaj de la homsocio en daùro de mil jaroj okazis sub la signo de tiuj du principoj : 
unuflanke, la lukto por regpovo, por konkero, konservo kaj plivastigo de la privilegioj kaj materialaj havaĵoj heriditaj aù konkeritaj, kiuj ĉiam firmigas la potencon de la homo super la homo ; aliflanke, la lukto por libereco mensa, materiala, religia, ekonomia aù sociala, persona aù kolektiva ; la lukto por defendo de la sociaj institucioj certigantaj ĉi tiun liberon, aù kontraù tiuj, kiuj  ĝin subpremas ; lukto de individuoj aù de pli-malpli gravaj grupoj, organizoj, klasoj, socioj kaj de tutaj civilizacioj, kiuj ofte pereis, aù saviĝis, sed eltenante la pereigajn konsekvencojn de tiu lukto.
}
\para{Du fundamentaj tendencoj karakterizas ĉiujn homosociojn de antikva Orienta epoko ĝis de hodiaùa tempo. La unua estas la popola, tiu de vastaj homamasoj, kiuj kreas riĉaĵojn, tiu de saĝuloj, poetoj, pentristoj, skulptistoj, arkitektoj, ktp. Ĝiaj  fruktoj estas popolkutimoj, sur kiuj masoniĝas ĉiuj sociaj institucioj : la klanoj, kampara komunumo, civitoj, folkloro, la saĝeco de la jarcentoj kaj la tuta spirita kaj materiala kulturo, kiu modlas la fizionomion de l'tero. En tiu tendenco trovas sian originon la senestrecaj ideoj. La alia tendenco estas la kontraùpopola -- tiu de la magiistoj, orakoloj kaj pastroj, de la nobeloj, kiuj konservas kaj transdonas la leĝojn kaj malnovepokajn tradiciojn, de la militestroj posedantoj de la armil- kaj venk- sekretoj. El tiu tendenco nutriĝas ĉiu aùtoritata ideo.}
\para{Tiel, la senestreca penso estas daùriganto de la popolaj tradicioj ; senestreca filozofio estas filozofio de la popola kutimjuro kaj natura rajto ; dum aùtoritata penso estas heredanto de la perfort-rajto, de la ruzulo, de la leĝ-admiranto, afero de la privilegiaj malplimultoj.}
\para{Sed ankaù la kutimjuro, la popolaj tradicioj suferas malnoviĝon, ridiĝon, kaj ili transformiĝas ne malofte en neadon de la progreso, en obstaklo de la evoluo, al la kreskanta spirita kulturo, al la kreskanta liberneceso. Tiam venas la protesto, la ribelo kontraù malnoveco, konservatismo, mumigo. Tiu ribelo estas persona aù kolektiva. Tamen ne ĉiam la ribeluloj estas portantoj de io esence nova. Iuj el ilis deziras purigadon de la kutimoj kaj institucioj por certigi pli grandan efikon en la konservado de la individuaj kaj sociaj interesoj. Ili strebas forigi ĉiujn malhelpojn for de la vojo al evoluo kaj libero. ; ili strebas neniigi naskinĝantajn formojn de regpovo, kaj tial ili estas revoluciaj. Meze inter ili naskiĝas anarkiaj ideoj. Aliaj volas reformi la malnovajn kutimojn kaj instituciojn por de ĉi-lastaj plene ``liberigi'' kaj trudi personan aùtoritaton. Inter ili troviĝas la antaùuloj de ĉiuj \qui{Hitler}-oj kaj \qui{Stalin}-oj -- de ĉiuj diktatoroj.}
\para{La liberecana idealo ĉerpas sian spiritnutraĵojn -- nukon de la eterna kaj neestingebla impeto de la popoloj al libereco kaj sendependeco -- el la persona aù kolektiva ribelo kontraù ĉia subpremado, superrego, kontraù ĉia privilegio, kiu ĉiam transformiĝas en superregon. lLa senestreca penso enradikiĝas en la ribelo, en la protesto, en la kritiko, en la neado de ĉio malnova kaj eksvalida, kiuj enkarniĝas en Satano laù la bibliaj legendoj, en Prometeo laù la greka mitologio. Ties portantoj estas la ĉiuepokaj sektoj, kiuj aperas, kiel tipa esprimo de la  spirita ribelo. Ĝiaj martiroj estas la ĉiuepokaj herezuloj, bruligitaj sur brulŝtiparoj aù vivante palisumataj. La anarkiisma penso manifestiĝas en definitiva formo en la mondkoncepto kaj agado de la bogomilanoj, albigensoj, husanoj, anabaptistoj.}
\para{Reprezentantoj de senestreco estas plebanoj, kiuj en antikva Romo unua-foje en la historio efektivigas la ĝeneralan strikon kaj ĉesigas ĉian vivon, forlasinte la urbon, disiĝinte de la patricioj, tiel devigante ĉi-lastaj cedi. }
\para{Senestreco en la historio estas reprezentita per agado de la kamparanaj insurekcioj kontraù servut-reĝimo, per la sekcioj de la granda franca revolucio, en la julia revolucio de 1830, en la februara de 1848, malgraù tio, ke ĝi havas ankoraù la formon de la nekoncia senestreca agado. Elementojn de ĝi ni ankaù havas en la Pariza Komuno\index{Pariza Komuno}, jam esprimo de taktiko lumigita de la ideoj de \qui{Prudono} kaj \qui{Bakunin}.}
\para{Alia fonto de la senetrecaj ideoj estas la filozofio de saĝuloj el diversaj epokoj spegulintaj la popolajn aspiradojn :\qui{Lao Tse} el antikva ĉinio ; la egipta gnostikulo \qui{Karpokrato} el Aleksandrio: \qui{Antigon} kaj \qui{Zenon} el antikva Grekujo ; verkistoj kaj pensuloj el pli nova tempo : \qui{Rabelezo}, \qui{Fenelon}, \qui{La Boesi} ; enciklopediistoj el la 18a jarcento kaj speciale \qui{Diderot} kaj ceteraj.}
\para{Tria fonto de la senestrecaj ideoj estas la Scienco per ties atingaĵoj kaj malkovroj.}
\para{Malgraù ĉio, ĉiuj ĉi fonto ne sufiĉas por vivigi bonkonstruitan doktrinon pri socia reorganizado. La teknika, ekonomia, materiala kaj spirita progreso en la fino de la 18a jarcento kaj dum la 19a venigis sur la mondscenejon novan kaj ĉie pli kaj pli klare formiĝantan kaj ekkonciantan klason. La proletaro kaj la luktoj, kiuj ĝi kvidis dum la tuta unua duono de la 19a j. finludis decidigan rolon de akuŝisto de la senestreca mondkoncepto.}
\para{Unua esprimanto de la sensestreca doktrino (1791) estas William \qui{Godwin} (1756-1836) en la libro : ``Esploroj pri la justeco en politiko kaj ĝia influo sur la virto kaj universala feloĉo'' ; dua estas \qui{Prudono} (1809-1865) en la libro : ``Kio estas proprieto ,'' aperinta en 1840.}
\para{Sed la formado de donita ideo en iun doktrinon ankoraù ne estas sufiĉa, por ke oni kalkulu kun ties realiĝo. Necesas organizado, necesas movado. La rektan originon de la senestreca movado oni devas serĉi en la Unua Internacio \index{Unua Internacio} kaj pli precize en la Parisa Komuno \index{Pariza Komuno}. Kreinto, organizinto kaj plej bona esprimanto de tiu movado aperas Mikael \qui{Bakunin} (1814-1876). Ĝis li, anarkiismo estas nur iu filozofia tendanco, plu literatura manifestiĝo ol senpera sociala agado. Per \qui{Bakunin} anarkiismo aliformiĝas en revolucian socialismon, en social-revolucian movadon.}
\para{Kaj ĉi tie, lastloke laù vico en la traktaĵo, sed ne laù graveco, venas la rolo de la teoriistaj. Ni ofte jam diris, kaj nun subtenas, ke senestreco ne estas laboratoria produkto ; sed tio ne signifas, ke ĝi estas frukto de mamekĝermo, ke la rolo de la teoriistoj estas sensignifa. De longtempe la ideoj ŝvebas en la atmosfero, ili manifestiĝas, ili montras sian ĉeeston, sed de tiam, ĝis kiam ili trovos iun belaranĝitan formon, ĝis kiam ili trovos siajn profundajn esprimantojn, ilia influo ne povas stabiliĝi.\\}
\para{En la posta ĉapitrovico, ni traktos la bazajn elementojn, kiuj karakterizas la ideologion, taktikon kaj konstruan programon de senestreco. Estos trarigardataj jenaj problemoj : libereco, solidareco, egaleco, justeco, libera iniciato, libera kontrakto, federismo, aùtoritato, ordonpovo, perforto, ekspluatado, milito, centralizismo, ŝtato, proprieto, eklezio, parlamento, leĝo, tribunalo, mallibero, polico, armeo, imposto, mono, lernejo kaj scienco, arto, gazetaro kaj radio, organizado, striko, bojkoto, sabotado, ribelo, insurekcio kaj revolucio, libera komunumo kaj ferderacio, produktorganizado, distribuo, diversaj unuiĝoj.}