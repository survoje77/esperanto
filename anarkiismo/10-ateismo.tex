\section*{Ateismo}
\addcontentsline{toc}{section}{Ateismo}
\indent 

\para{Paralele kun la senestreco, t.e. kun neado de la aŭtoritato kaj regpovo, iras la neado de Dio, de la kredo je Dio -- la ateismo, kiu ankaŭ presentas unu el la principoj de senestreco starigitaj laŭ la neadvojo. La ideo mem pri Dio estas plena kaj senkondiĉa neado de la fundamentaj principoj de anarkiismo, de libera homa socio : libero, solidareco, egaleco, justeco, libera iniciativo, ktp... Sekve, eltirante el la vivo kaj el la historio de la homa socio tiujn bazajn principojn, la anarkiismo neeviteble, logike refutas Dion kaj la tutan tiun sistemon de kredoj, superstiĉoj kaj historiaj surtavoliĝoj el stultaĵoj kaj malklero, kiun oni nomas religio.}

\para{Neniu alia socifilozofia doktrino atribuis tiun grandegan gravecon al la religio en la sklavigo de la homo, kaj estas tiel fundamente studinta ĉiujn flankojn de la kredo je Dio, kiom la anarkiismo. Riĉa estas la anarkiisma literaturo, traktanta tiun demandon. Ne estas eĉ unu teoristo, kiu estas ne okupiĝinta pri ĝi: \qui{Godwin}, \qui{Prudon}, \qui{Kropotkin}, \qui{Faure}, k.a. Estas precipe \qui{Bakunin}, kiu profunde penetris en tiun problemon eternan por sufiĉe klera inteligento, kaj kiu fulmbrile esprimis la anarkiisman penson sur tiu kampo.}

\para{«La ideo pri Dio, skribas \qui{Bakunin}, kuntrenas la kripligon de la racio kaj falsas la homan justecon. Ĝi estas la plej decida neado de la homlibero kaj neeviteble kondukas al sklavigo de la homoj same teorie kiel praktike… La religioj disbatas la homajn fierecon kaj indecon, ĉar ili favoras la rampantojn kaj humilulojn. Ili sufokas en la koroj de l’ popoloj la senton de homfrateco kaj ilin plenigas per dieca krueleco.}

\para{«Ĉiuj religioj estas kruelaj, ĉiuj konstruiĝas sur sango; ĉar ĉiuj baziĝas ĉefe sur la ideo pri oferdono.»}

\para{Aliloke \qui{Bakunin} skribas: «La religio estas kolektiva frenezeco, kiu montriĝas des pli potenca, ju pli ĝi aperas tradicia, kaj ju pli ĝia origino perdiĝas en senlime mal–proksima antikveco. Kiel kolektiva frenezeco ĝi penetris same en la socion, kiel en la privatan vivon de la popolo; ĝi enradikiĝis en la socio, iĝis, se tiel diri, kolektiva animo kaj penso. Ĉia homo jam de sia naskiĝo estas ĉirkaŭprenita de ĝi, li ĝin englutas kun sia patrina lakto, ĝin ensorbas kun ĉio, kion li aŭdas kaj vidas. Li estas tiom nutrita per ĝi, tiom venenita, ĝi tiom penetras en lian estaĵon, ke pli malfrue, kiom ajn potenca estos lia spirito, li bezonos grandajn fortostreĉojn por liberiĝi de ĝi, kaj en tio li eĉ neniam plene sukcesos. Malbonaŭgura historia fantomo kreita de la imago de la primitivaj homoj de kvar aŭ kvin miljaroj ĝi pezas sur la sciado, sur la libereco, sur la humaneco, sur la vivo.» Se antaŭ 85 jaroj, kiam \qui{Bakunin} donis siajn pezajn batojn sur la teologion, estis multaj homoj, kiuj demandis sin, kial estas tiom necese okupiĝi pri Dio kaj la religio (lasu la homojn kredi je kio ajn ili deziras, okupiĝu nur pri la socia problemo) ankoraŭ pli granda devas esti la nombro de tiaj homoj hodiaŭ ĉe la granda progreso de la socia kaj scienca penso… \qui{Bakunin} tamen ne eraris. Lia antaŭvido rilate al la pereiga influo de la religio validas ankaŭ por hodiaŭ: «la di–regno en la ĉielo transformiĝas en regnon –agnoskitan aŭ maskitan– de la skurgo kaj de la ekspluatado de la laboro de sklavigitaj amasoj, tie ĉi sur la tero… Unu mastro en la ĉielo sufiĉas por krei milojn da mastroj sur la tero.»}

\para{Ĉu ni ne vidas ankaŭ hodiaŭ, paralele kun plifortiĝo de la sklaveco, relativan renaskiĝon de la religio, kiel en Oriento, tiel en Okcidento? Ĉu ni ne vidas ankaŭ hodiaŭ, ke la religio iĝas, speciale sub la tiranaj reĝimoj, rifuĝ–ejo por la senkuraĝigitoj, iluzia konsolo de la humiligitoj kaj senpersonigitoj? Ĉu ni ne vidas aliflanke, ke tiu «opio por la animoj» restas unu el la neanstataŭeblaj kaj nepre necesaj rimedoj por firmiĝo de la ordonpovo? Kiel alie oni povas klarigi la oficialan agnoskon, toleradon kaj eĉ materialan helpon al la eklezio fare de la reĝimoj, kiuj nomas sin «socialistaj»? Kiel oni povas alimaniere klarigi la fakton, ke homoj kaj partioj, kiuj volas, sin prezenti kiel tre soci–progresemajn, kiuj mem ne estas religiaj kaj kredas je nenio krom je sia amata ordonpovo, «ne ĉesas paroli pri religio kaj eklezio, kaj kun farisea pihumileco serĉas la aliancon de \emph{paŝtistoj de la animoj} »? Ĉion ĉi bone penetris \qui{Bakunin} por veni al decidiga konkludo: «Se Dio efektive ekzistus, oni devus neniigi lin.» Kaj fakte, se oni analizas la religian predikon de kiu ajn konfesio, oni vidas ke la ideo pri Dio kaj kredo je Dio estas plena neado de la homo, de lia konscio pri libereco, indeco, solidareco, egaleco, iniciativo… Eĉ la amo, kiun oni konsideras kvazaŭ «registrita marko» por ĉiuj religioj, ne havas homan enhavon. Laŭ la religio Dio estas ĉio, homo nenio, ĉio apartenas al Dio, ĉio devenas de Dio. Ĉion oni ŝuldas al Dio. La unua devo de la kredanto estas agnoski sin neniaĵo, surgenuiĝi antaŭ Dio, –kaj antaŭ liaj surteraj reprezentantoj, kompreneble– ankaŭ deklari: «mi estas nenio, al ci mi apartenas, mia Dio, je ciaj ordonoj mi estas». Kian liberon oni povas atendi de homo, kiu memvole kaj kun humileco rezignas, kaj fordonas sin en la manoj de Dio kaj ties surteraj reprezentantoj? Ĉu oni povas atendi de li liberan iniciaton? Aŭ ĉu solidareco, justeco, egaleco, eĉ ankaŭ amo, kiam tio por li ne estas principoj, ne estas liaj sentoj propraj, ne estas parto de li mem, de lia ekzistaĵo? Eĉ la amo, kiun eble la kredanto sin konsideras devigata nur manifesti, devenas de Dio; oni agas en ties nomo kaj laŭ ties ordono. Ĉar Dio estas fore kaj alte, tial plej ofte liaj anstataŭantoj decidas al kiu, kiel kaj kiagrade oni devas manifesti solidarecon, justecon, amon. Pri egaleco eĉ ne povas temi; ĉu ja estas pli granda neegaleco en tio? El tio neeviteble sekvas hierarkiaj gradoj –neado de la nocio mem pri egaleco. Sur tiu psikologia grundo –psikologio de sklavo– bone sterkata per grasa malklero, sisteme tenata, facile estas semi kaj kulturi ĉiuspecajn superstiĉojn, oportunajn sentojn al tiuj, kiuj fakte komandas la agojn, la konduton, la moralon de la kredantoj –«la spiritaj paŝtistoj», kiuj plej ofte iras man’ en mano kun la laikaj paŝtistoj de homgregoj.}

\para{Ĉu do estas mirige, ke la kristanismo, ekzemple, kiu historie montriĝas religio de la humiligitoj kaj ofenditoj, de la malriĉuloj, malestimitoj, forĵetitoj kaj servutigitoj, religio de la amo kaj egaleco, estas kreinta la inkvizicion kaj pleniginta la homhistorion per tiom da sango, flamigante kaj kuraĝigante ĉion, kio estas neado de la libero, frateco, egaleco, justeco, amo? Laŭ la enhavo, kiun donis al ĝi KRISTO kaj liaj disĉiploj, la kristanismo estas internacia religio, ĉiuj homoj sur la tero estas por KRISTO fratoj. Eŭropo, Ameriko, Aŭstralio, parto el Afriko estas kristanaj. Kaj ĝuste tiu kristana mondo en la daŭro de jarcentoj intermortigas sin, ĉiam en la nomo de la Dio de la amo. Kaj la religiaj militoj, kaj la sektoj? Ĉu estas pli granda semanto de malamo ol la religio? Ĉu penvaloras, ke ni konstatu aferojn de ĉiuj sciatajn?}

\para{Estas klare, ke la religio kontraŭdiras al la plej elemen–taj kaj naturaj sentoj kaj principoj pri la homrilatoj. Ĝi estas en konflikto ankaŭ kun la scienca penso; ĝi estas refutita ankaŭ de la scienco.}

\para{Kiel oni povas klarigi la fakton, ke tiu tiel negativa kaj por la homaro malutila ideo pri Dio povis naskiĝi kaj pluiĝi dum miljaroj, teni eĉ hodiaŭ milionojn da homoj en spirita sklaveco –nepre necesa kondiĉo por ĉia alia sklaveco?}

\para{Supernatura laŭ sia esenco, la ideo pri Dio, la kredo je nevidebla kaj supernatura estaĵo estas plene natura, materiala, kiel ĉiu alia ideo, laŭ origino. Ĝi enradikiĝas en la homa, aŭ pli ĝuste besta, naturo de l’ homo. Tion ĉi oni povus konsideri kiel ofendon, insulton por la patroj de la eklezio, kiuj atribuas al la religio ĉielan kaj dian originon, sed ĝi estas pura vero.}

\para{La religio radikiĝas en la natura sento de timo, propra al ĉiuj bestoj, kaj nedisigebla parto el la instinkto de memkonservado. Tiusence, laŭ esenco, la religia sento enĝerma ankaŭ ne estas fremda al la bestoj. Sed kiel ideo, kiel koncepto, la religio estas frukto de la limigita kaj malklera homa cerbo; sekve ĝi estas homa, sed ne besta produkto. Ĝi estas eble la unua, certe ekstreme malperfekta homa manifestiĝo, kiu metas la limon inter la besta kaj homa mondoj, ĉar ĝi estas kreita de la penso –unua distinga kvalito homa– provanta klarigi la fenomenojn.}

\para{La tuta universo, la tuta videbla kaj nevidebla mondo, la tuta senviva kaj viva naturo prezentas tutecon, univer–salan solidarecon, konstante transformiĝantan fortorezultanton, eternan fluadon de la materio –universalan vivon. Tiel estas la koncepto al kiu atingis la hodiaŭa materialisma scienca penso –frukto de multjarcentaj observoj, serĉadoj, pristudoj. Sed ĉu estus povinta la primitiva homo, per siaj ekstreme limigitaj scioj kaj imagoj, formi tian koncepton? Nature, ne. Sed li, sen ia dubo, sentis la ekziston de tiu universala solidareco, kies konsista parteto li mem estas, same kiel sentas ankaŭ la bestoj. Diference de ĉi-lastaj, al kiuj ligas lin la komuna instinkta timo pro la naturaj elementoj, fortoj kaj fenomenoj, li provis klarigi la ĉirkaŭan mondon, kaj malkapabla fari tion la primitiva homo venis al animigo kaj adorado de la neklarigeblaj fortoj kaj fenomenoj. Tiel ekaperas la religio, kiu ankaŭ montriĝas la primara filozofio de l’ homo. La plua tuta historio de la religio, en ĉiuj ĝiaj etapoj kaj formoj ĝis la monoteismo, kiu metas la efektivajn bazojn de la religio en la nuntempa senco kun ĝia socia rolo, estas historio de malrapide retiriĝanta malklereco sub la lumo de ŝajne tiel malrapida progresanta sciado.}

\para{La volo kaj la konscio iom post iom anstataŭas la timon, la malklerecon, kaj la religio retiriĝas antaŭ la racio. Kiu kredas, tiu nescias –kaj tiu ne scias, kial li kredas; kiu scias, tiu ne kredas. Tiel la scienco iom post iom okupas la lokon de la religio, kaj la filozofio –sintezo de la scienca penso– entombigas la ideon pri Dio. La homo, venkinta la timon kaj malklerecon, iras al efektivigo de sia libero –la kulmina punkto de la humanigo de la besto–homo. Nek Dio, nek mastro –kio signifas liberan homon.}