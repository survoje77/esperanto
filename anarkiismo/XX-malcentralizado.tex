\section*{Malcentralizado}
\addcontentsline{toc}{section}{Malcentralizado}
\indent 

«Ĉe la centralizado vi havas apopleksion en la centro kaj paralizo en la ekstremajoj»

LAMENE (Lamennais)

«Historio instruas al ni, ke troa centralizado en malgran–da skalo samkiel en granda skalo estas malutilaj; aŭtonomion oni devas agnoski de ĉia natura unuo: ekononia, spirita kaj eĉ nacia».

T.G. MASARIK

«Ĉia minacata povo sin turnas al centralizado».

Jean–François GRAVIER

En la ordinaraj, kurantaj, ĉiutagaj eldiroj oni ofte eraras, identigante federismon kaj malcentralizadon tial, ke ili fakte estas proksimaj kaj reciproke sin kondiĉas. Tamen, la neceso de precizeco devigas nin dislimi ĉi tiujn du nociojn respondantajn al du principoj. Federismo estas maniero de funkciado, dum la malcentralizadon karakterizas strukturo de donita organizaĵo. Sekve, la principo de federismo, kiun ni jam traktis, malgraŭ tio, ke ĝi postulas kaj kondiĉas la malcentralizadon, tute ne elĉerpas ĝin. La malcentralizado same presentas principon, kiu radikale distingas anarkiismon disde ĉiuj aŭtoritataj konceptoj. Tiu ĉi principo kontraŭstaras al centralizado karakterizanta ĉiujn organizojn kaj sociojn, kiuj konstruiĝas sur la aŭtoritataj principoj.

La akcelita kapitalisma disvolviĝo, sekvinta la vastan aplikon de la vapormaŝino en la komenco de la 19a jarcento, tenis la atenton de la sociologoj el tiu epoko kun centralismatendenco, kiu konturiĝis kiel ununura kaj universala en la ekonomio, kaj maskis, kiel ofte okazas en la socia vivo kiam ĝi ne estas sufiĉe lumigita ideologie, la ekziston de ĉia alia tendenco. Preskaŭ ĉiuj socialistoj de la tempo 1848a akceptis la centralizisman tendencon kiel neeviteblan fenomenon de la kapitalisma evoluo. Ankaŭ MARKS nenion novan alportis tiurilate; li per sia aŭtoritato nur trudis koncepton, kiun multaj akceptis. La fakto, ke la tekniko trudas certan koncentradon al la entreprenoj –ĉekio unuj el ili malaperas profite al pligrandiĝo de la aliaj– estis tiom evidenta, ke sufiĉis doni al ĝi duonsciencan argumenton, por ke ĝin akceptu la malmulte kritikema mezcerbo, kiel sociologian leĝon. Kaj de tiam ĝis hodiaŭ, kaj kleraj kaj malkleraj, kaj malgrandaj kaj grandaj, papage ripetas banalajn verojn: «la granda industrio englutas la malgrandan», «la granda fiŝo formanĝas la etan», ktp. Unu jarcento elfluis de tiam, kaj la sablona analizo de la ekonomia evoluo plurestas por multaj sur la punkto de glaciiĝo de la Marksa dogma sociologio.

Tiu fatala eraro, kiel ĝenerale ĉia eraro, koncernante vastan rondon da homoj, tiom obsedis la malseriozajn sociologojn kaj ekonomikistojn, ke, kiam KROPOTKIN en 1898 publikigis anglalingve sian libron Kamparoj, fabrikoj kaj metiejoj,poste tradukita en ĉiujn lingvojn, la pensoj, kiujn li tie elmetadis estis konsiderataj de multaj kiel herezon en la ekonomika scienco; por aliaj ili montris ĝis kie povas iri naivaj sonĝoj de simplulo sen ia ajn scienca valoro, kaj ĉe triaj estiĝis nur ŝercmokon. En la lasta kategorio estis kompreneble sur la unua loko la tre memfidaj «sciencaj socialistoj». La centralizisma narkoto tiom ebriigis la cerbojn, ke ili estis nekapablaj vidi la plej okulfrapajn fenomenojn, eĉ kiam ĉi–lastaj estis argumentitaj de efektiva scienculo, kiel estis KROPOTKIN.

Malpli ol duona jarcento sufiĉis por ke oni vidu realigitaj la verojn rimarkigitajn de KROPOTKIN, kaj por ke ĉi–lastaj estu ŝtelume akceptitaj eĉ de la hodiaŭaj bolŝevis–toj. Bedaŭrinde, la oficiala marksisma ekonomia scienco ankoraŭ sin ŝajnigas neaŭdanta, kaj ne havas la noblan kuraĝon konfesi sian eraron kaj la netakseblan meriton de KROPOTKIN, kies rekomendoj en tiu sfero trovas ĉiam pli vastan aplikon, kaj eĉ iĝas bazo de la oficiala ekonomiko en multaj landoj. Ne estas hazarde, kompreneble, ĉar laŭ la samaj sociologoj kaj ekonomikistoj –por kiuj la socia kaj ĉia alia disvolviĝo en la homrilatoj kondiĉiĝas de la ekonomia kaj eĉ pli precize de la teknika disvolviĝo– la ekonomia centralizado devis esti sekvita de la socia centralizado, kies personigo estas la moderna ŝtato. Kaj ĝuste male por KROPOTKIN kaj por ĉiuj anarkiistoj la tendenco al ekonomia malcentralizado respondis al la ĉe la homo natura strebo al libereco kaj sendependeco; kaj ĝi aperis apliko en la donita okazo nur en unu fako –la ekonomia– de principo, kiu egale validas por la strukturo de la socio en ties tuteco. La malcentralizado estas nedisigebla en sia plena aplikado de la anarkiisma koncepto pri la homo kaj la socio. Tial la aŭtoritatuloj volas silente preterpasi la faktojn, profitante, laŭ neevitebla neceso, de la parta apliko de principo respondanta al la naturaj postuloj de la vivo.

Sed ni ekzamenu la problemon pli precize. La centralizado, kiel strebado kaj kiel principo, akompanis ĉian aŭtoritaton dum la tuta historia evoluado de la homaro; ĝi tamen manifestiĝas speciale forte ĉe la plena diskresko de la ŝtato –dum la periodo ekde la disfalo de la romia imperio ĝis hodiaŭa tempo– ĉe la apero kaj konstruado de la nuntempa moderna ŝtato nacia post la 16a jarcento. Sur la ekonomia tereno, kiu pli speciale nin interesas ĉi tie, la centralizado koincidas kun la apero kaj rapida disvolviĝo de la kapitalismo, ligita kun la malkovro en 1767 de la vapormaŝino, kies «erao» daŭras proksimume ĝis 1850, kiam grandaj sciencaj malkovroj kontribuas al la naskiĝo de la elektro, kiu per sia vasta apliko inter 1875 kaj 1900 disbatas la fundamentojn de la industria centralizado. MARKS kaj ĉiuj, kiuj ĝis hodiaŭ ne ĉesis dividi la koncepton, laŭ kiu la centralizado estas neevitebla fenomeno de la teknika kaj ekonomia disvolviĝo, blindigitaj de la superanta tendenco dum tiu unua periodo nur unuflanke vidis kaj prezentis la nekontesteblan tendencon de la industria centralizado dum la vapormaŝina epoko. La karbodonaj tavoloj aperis determinanta faktoro kaj komanda centro. Fakte ankaŭ tie ĉi la geografia faktoro kondiĉas la rolon de la tekniko. Anglujo kaj poste Germanujo en la imago de la tiamaj ekonomikistoj en harmonio kun la interesoj de la supreniranta kapitalismo, prezentiĝis kiel nediskuteblaj centroj de la tutmonda industrio dank' al la ĉeesto de karbodonaj tavoloj, la ununura fonto de energio por moderna industrio. Tiel sur la nacia plano la regionoj riĉaj je ŝtonkarboj devis «fatale» iĝi komandaj centroj de la industria disvolviĝo, kaj la koncernaj landoj same tiel «fatale» devis transformiĝi en grandajn uzinojn por la cetera mondo, kondamnita esti liveranto de krudmaterialoj kaj agrokul–turaj produktoj kaj sklavigita konsumanto de iliaj indus–triaj produktoj. KROPOTKIN kiel vera, serioza kaj antaŭ sia konscienco respondeca scienculo, profunde trastudis same la eraon de ŝtonkarba hegemonio, kiel la sekvintan periodon de la liberiga elektro, ne kontentigante per la nesufiĉaj statistikoj tiamaj, sed li mem, aŭ kunlaborantoj–korespondantoj, surloke kontrolante la faktojn. Li laŭtigis protestan voĉon kontraŭ la eraro kaj pruvis, ke paralele kun la centralizado en la industrio, efektiviĝas ankaŭ malcentralizado, kiu montris nenian signojn de plena engluto kaj forigo de la etindustrio–kontraŭe, ĉi–lasta ofte aperis eĉ pli vivkapabla. Li emfazis same, ke sur la internacia plano la geografia malcentralizado manifestiĝas ĉiam pli kaj pli potence, ke la popoloj strebas disvolvi sian propran industrion. Tiu ĉi tendenco, laŭ li, unisone kun la vera idealo de la homo, kiu estas proksimigi la fabrikon al la kamparo, la sciencan laboratorion al la metiejo, por atingi integradon de la intelekta, fizika aŭ permana laboro, kiu, ununura, povas garantii la homan harmonian disvolviĝon, bazon de ĉiuflanka progreso. Kaj la vivo, kiel ni vidos, konfirmis tiujn konceptojn.

La centralizado geografia kaj teknika de la industrio estas ankaŭ fakto, kiun KROPOTKIN neniam neis. Konvenas rimarki tie ĉi, tamen, ke ankaŭ la ekspliko de tiu ĉi fakto estas unulatera flanke de la marksistoj. La potenca industriigo kiel la centralizado en la industrio estis decidige helpata de la ŝtata interveno, kiu ruinigis la agrokulturon kaj perforte proletigis la kamparanojn en Anglujo por krei la necesan manlaboron. Plej konvinka kaj okulfrapa estas la administra–ŝtata rolo en la mortiga centralizado, artefarite enplantita en Francujo, kaj kiun hodiaŭ unuanime kontestas la plej elstaraj intelektuloj en tiu ĉi lando. Ĉiuj sociologoj, ekonomikistoj kaj historiistoj hodiaŭ konfesas, ke tiu ĉi sufokanta centralizado, kontraŭ kiu oni gvidas jam organizitan kaj laŭplanan lukton, radikiĝas en la administra politiko de la francaj ŝtatistoj jam en la tempo de RICHELIEU, COLBERT, la Granda Franca Revolucio, NAPOLEONO la Unua kaj NAPOLEONO la Tria. Por tiu politiko Parizo estas Francujo. Ĉio devis centraliziĝi tien. Eĉ rigardante la fervojreton sur la landmapo, oni komprenos ĉi tiun politikon, kiu neniel estas diktata de ekonomia–geografiaj kaj teknikaj konsideroj. De la plej superaj postenoj oni hodiaŭ alarmas kontraŭ tiu ĉi centralizado. Specialaj oficoj kaj institucioj hodiaŭ estas komisiitaj malcentralizi. Oficiala komuniko antaŭ kelkdudek jaroj publikigita asertas, ke 6.043.693 personoj sufokiĝas en radiuso de 25 kilometroj de Parizo kaj, ke de la trideka kilometro komenciĝas «la franca dezerto».

La monstra centralizado maltrankviligas ne nur Francujon, sed ankaŭ ciujn ceterajn industriajn landojn, kie la grand–urboj estas veraj frenezulejoj kaj menaĝerioj, ĉiutage englutantaj milojn da viktimoj nur pro la trafikaj malhelpoj, sendepende de ĉiuj afliktaj kaj domaĝaj konsekvencoj kaj de plikariĝo de la ensocia vivo. Anglujo kiel unua reagas. Usono ankaŭ ne malfruiĝis por entrepreni antaŭzorgojn kontraŭ la centralizado. La demando estas tiom vasta, la tiurilataj donitaĵoj estas tiom riĉaj, ke tutan libron oni povus verki kiel kompletigon kaj konfirmon de la antaŭvidoj de KROPOTKIN1. Eĉ la Stalina Ruslando ne restis, almenaŭ laŭ vortoj, malantaŭe. La sesa kvinjara plano antaŭvidas: «certigi plibonigon de la distribuado de la produktantaj fortoj, alproksimiĝon de la industrio ĝis la krudmaterialaj fontoj, ĝis la termo-elektrocentralaj fontoj kaj ĝis la konsumantaj regionoj. Certigi la prudentan specialiĝon kaj la konpleksan evoluon de la ekonomio en ekonomiaj regionoj, konsidere al la plej kompleta kaj efektiva utiligo de la naturaj riĉofontoj kaj manlaboran rimedoj cele al plikreskado de la produktiveco de la socia laboro.» Tio ĉi estas nek pli nek malpli ol la malcentralizado en la industrio, antaŭvidita kaj rekomendita de KROPOTKIN.

Ni ne intencas tie ĉi esprimi nian opinion pri la antaŭ–zorgoj faritaj tie ĉi kaj tie por efektivigi la necesan malcentralizadon, nek fari tiurilate antaŭvidojn. Tiu ĉi tasko ne havas lokon en nia temo. Ni povas nur diri, ke ni neniel iluziiĝas rilate al la ebloj realigi tian malcentralizadon –antaŭviditan kaj defendatan de la anarkiistoj– en la kadroj de ŝtatkapitalisma socio. Nia celo estas emfazi, ke unu el la fundamentaj principoj de la anarkiisma sociologio –la malcentralizado– estas entute konfirmita de la nuntempa socia evoluo. La morgaŭa apliko de la atomenergio, post tiu de la elektro, efektivigos teknikrilate kompletan realiĝon de tiu ĉi principo.