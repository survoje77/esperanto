\section*{Federismo}
\addcontentsline{toc}{section}{Federismo}
\indent 

\para{La pridiskutitaj unuaj kvin principoj de la senestreco : libero, solidareco, egaleco, justeco kaj libera iniciativo trovas sian kunordiĝon, harmoniiĝon en la libera kontrakto, principo, kiu prezentas juran bazon de la senestreco kaj la formon, en kiun enfluas la homaj rilatoj alprenante karakteron de sistemo, de organismo.}

\para{Federismo ja prezentas tiun gravan principon, kiu donas vivon al tiu ĉi organismo. Ĝi prezentas la dinamikon, la manieron funkciigi la socion, certigantaj la liberon, solidarecon, egalecon, justecon, la iniciaton personan kaj kolektivan kaj la liberajn interkonsentojn inter la homoj vivantaj en socio. La enhavo determinas la formon. La formo sufokas la enhavon, se ĝi ne kongruas kun ĝi. La enhavo disbatas la formon, se ĉi–lasta kontraŭas al ĝi, kaj ĝi serĉas sian formon. Se ĉi tiuj bazaj principoj de senestreco (kies manifestiĝojn ni trovas en ĉiuj socioj kaj tempoj kaj eĉ ankaŭ en la animala regno) ne povis realiĝi ĝis hodiaŭ en sia pleneco, estas pro tio ke la formo kaj funkciado de la sociaj rilatoj kontraŭis al ili kaj ilin sufokadis. De tio, oni klare komprenas, kiom granda kaj ekskluziva estas la graveco de la federismo kiel principo de konstruo kaj funkcimaniero de la socio.}

\para{La principo de federismo pliampleksiĝas kaj trovas apli–kon eĉ en la neorganika mondo, alprenante karakteron de universala principo en la naturo. Kiel \qui{Kropotkin} majstre tion montris kaj pruvis, federismo en pli nova tempo iĝis baza principo de la homa scio, principo de la modernaj scienco kaj filozofio. Viktimoj kaj sklavoj de sia propra elpensaĵo —Dio— el nememoreblaj tempoj la homoj vivis kun erara ideo pri la kosmo, vidante en ĉio kaj ĉie simpligitan skemon de «centron kaj periferio«. La centro, la sumo, la granda dimensio estas la fonto de l’ forto, energio kaj vivo. La periferio estas respeguliĝo. Ĉio komenciĝas ekde la centro kaj iras al la periferio. Ĉio ĉirkaŭiras la centron, subiĝas al, determinas sian pozicion laŭ, kaj dependas de la centro. En la imago pri kosmostrukturo la tero estis ĝis la 16–a jarcento konsiderata kiel la centro. Ĉio videbla en la Universo ĉirkaŭiris la teron, submetiĝis al ĝi kaj ekzistis nur por ĝi —la suno, luno, steloj, planedoj. Poste oni komprenis, kaj estis konstatite de la astronomoj, ke la tero prezentiĝas kiel nur sablero en la kosmo, kompare ekzemple kun la suno, ĉirkaŭ kiu ĝi fakte rondiras. Oni konstatis same, ke la suno ne estas ununura, ke multego da aliaj sunaj sistemoj ekzistas; kompare kun ili, ĝi prezentiĝas kiel nur sensignifa stelo. Sekvante ĉi tiun vojon de la disvastiĝo de siaj scioj kaj vizioj pri strukturo de la Universo, la astronomoj konstatis, ke ties senlimeco, samkiel nova principo de interrilatoj inter ties konsistpartoj, atingas al neado de la ideo pri centroj. La scienca serĉado forlasis siajn preferojn al la grandaj ensembloj kaj turnis sian rigardon al la konsistpartoj, vidante ĉiam pli kaj pli fore al la nemezureblaj eretoj. Tiel oni konstatis, ke la universala gravito ne prezentas skemon de «centro kaj periferio», sed ke ĝi estas rezultanto de senlima nombro da fortoj interagantaj kaj kontraŭagantaj, kie ĉiu parteto —eĉ la plej mikroskopa inter la miliardoj da nemezureblaj eretoj, kiuj plenigas la universon— havas sian lokon, gravecon kaj influon.}

\para{Tiu ĉi nova vizio pri la mondo, tiu ĉi rigardo super la aĵoj, tiu ĉi nova sinteno antaŭ la fenomenoj perceptitaj pere de niaj sensorganoj, larĝiĝis super ĉiuj sciencaj esploroj kaj etendiĝis super la tuta neorganika kaj organika materio, ampleksante ankaŭ la homsocion kaj nian internan mondon.}

\para{Ne plu estas hodiaŭ scienca disciplino, kiu ne sekvas tiun vojon de serĉado pri la konsistaj partoj, pri la unuo, nemezurebla ereto, kaj kiuj ne deziras trovi la konsistig–antojn eĉ de tiuj nemezureblaj eretoj. lu spiritostato ema al individuigo, malcentralizo kaj esploro pri «rezul–tantoj», pri dinamikaj, moveblaj rilatoj , karakterizas la tutan nuntempan sciencon, kaj naskas novan filozofion, en kiu la iama ideo pri «Centro» jam ŝanĝiĝis al nocio pri senlime nemezureblaj eretoj. En tio ĝuste konsistas la principo de federismo, principo kiu trovis la lastan sian konfirmon en la scienca sukceso diserigi la atomon, kaj tiel detrui la ideon pri «Centro».}

\para{Tiumaniere la principo de federismo —unu el la funda–mentaj principoj de la senestreco— iĝas universale agnoskita principo de la Universo kaj de la vivo. Sed la homaj ideoj estas ankoraŭ sklavaj de la centralizisma skemo, la homlingvaĵo, elforgita en la daŭro de la jarcentoj sub la subpremado de tiu skemo devenanta de la ideo pri Dio —la plej granda, la ununura centro, ĉefcentro, la komenco kaj la fino de la universala vivo— estas ankoraŭ tro malriĉa por esprimi en konvenaj terminoj la esencon mem de la federisma principo. Ni plu vidas nin devigataj uzi analogiojn kaj terminojn, kiuj, komprenitaj en sia laŭlitera senco, ne permesas al ni realigi plenan rompon kun la ideo pri centro, kaj de tie ankaŭ pri hierarkio. Tiel, ekzemple, dezirante prezenti federismon en pli simpla kaj pli komprenebla formo, ni ofte diras: «de sube al supre, de la periferio al la centro, ekde la privata afero al la ĝenerala ktp».}

\para{Kiel pli precize difini la principon de federismo? Ĝi estas la principo, kiu certigas la plenan manifestiĝon kaj la plenan disvolviĝon de la homa persono, manifestiĝo kaj disvolviĝo harmoniigitaj kun la plej alta evoluo de la memvola asocio en ĉiuj manifestiĝoj por ĉiuj celoj kaj por kontentigo de ĉiuj bezonoj. Tio signifas, ke ĉia iniciato devas starti de la aparta unuo–individuo en la grupo, kaj primara grupo aŭ unuiĝo en la ĝenerala kolektivo —provante kunordiĝi, unuiĝi kaj grupiĝi kun la sama iniciato de ĉia alia unuo, cele al pliiĝo de sia forto, larĝigante kaj pliigante siajn rimedojn.}

\para{Atinginte la plej plenan esprimon de la kolektiva harmo–niigo laŭ la linio de la praktikata funkcio aŭ en la terito–riaj limoj, kie la senpera partopreno de la unuo estas ebla, tiu iniciativo sen senpersoneciĝi aŭ forviŝiĝi en la kolektiva tuto, rezultanto de ĉiuj iniciativoj, transiras en pli larĝan teritorian stupon por harmoniiĝi kun la similaj iniciatoj en tiu larĝigita kadro kaj atingi la plenan kunordiĝon sur granda teritorio anticipe determinita aŭ akceptita (regiono, lando, kontinento aŭ la tuta terglobo). Atinginte la celitan plenan harmoniiĝon, la iniciato de la unuo –individuo aŭ grupo– ne nuliĝas, ne solviĝas en simpla aritmetika sumo, ne transformiĝas en komandantan kaj determinantan centron, sed en «rezultan–ton», por malkomponiĝi denove trairante la malan vojon, kaj veni ĝis la viveca, reala kaj dinamika unuo, kie la iniciato alprenas formon de memvola, konscia, libera devontigo, aŭ de devigo. Ne estas periferio, ne estas centro, estas nur sennombraj koordinatoj, eterna movado, manifestiĝo de milionoj da voloj, kies tuto, kolektivo ne estas adicio sed sistemo de permanentaj rilatoj, de konstanta movado de unuo al unuo —senlima konstelacio, ĉirkaŭiranta la senlimecon en kiu ĉiu stela polvero sekvas sian propran orbiton— rezultanto de agantaj kaj kontraŭagantaj fortoj kaj la universala orbito de la konstelacio, rezultanto mem de tiaj nenombreblaj fortoj.}

\para{Federismo aplikata en la homa socio estas tiu principo kiu, agnoskinte la liberon de la unuo–individuo unue, kolektivo poste —kiel fundamentan, precipan kaj absolute ne anstataŭeblan principon de l’ homa disvolviĝo— certigas la firmiĝon, diskreskon kaj plivastiĝon de la libero per la solidareco ebla dank’ al egaleco, justeco kaj aktiva, atentema iniciativo, ĉiuj —libero, solidareco, egaleco, justeco kaj krea iniciativo— jam proklamitaj, konscie akceptitaj kaj konfirmitaj en la libera kontrakto.}

\para{El kie estas eltirita la principo de federismo? Ĉu gi ekzistas en la vivo, en la homa socio, en la estinteco kaj nuntempo, aŭ ĉu ĝi estas nur principo pruvita kaj akceptita de la scienco kaj filozofio en ties serĉado de la eterna vero? Federismo havas siajn profundajn radikojn en la biologia bezono de la kunvivado, de la kuniĝo, de la kunordiĝo de la individuaj fortoj por certigi la memkonservon de la individuo kaj de la specio. Federismo en la historio de la homaro manifestiĝas paralele kun la sociemo. Ju pli forte kaj ju pli larĝe kaj potence manifestiĝas ĉi–lasta, des pli ebla estas la manifestiĝo de la unua, kaj inverse ju pli la federismo —nepre necesa kondiĉo por manifestiĝo de la homlibero— estas reala, des pli ebla estas la sociemo kaj kun ĝi la spirita kaj materiala progreso. Ankaŭ ĉi tie la liberaj mezepokaj civitoj, ofte prenitaj de ni por ekzemplo, montriĝas kiel kunvivantaro kaj kiel periodo el la homhistorio, dum kiu federismo trovis la plej larĝan aplikon, ĉiuj tiuj multnombraj kaj plej diversaj unuiĝoj interne de la sama civito, kiel ankaŭ inter la apartaj civitoj–respublikoj, kovrantaj la tutan mediteranean kaj okcidentan Eŭropon kaj parton el Ruslando, per diversaj unuiĝoj funkciantaj sur la bazo de la libera kontrakto kaj federisma principo estas la plej konvinka ekzemplo pri socia ekflorado materia kaj kultura, kiu estas ebla dank’ al la federismo.}

\para{Federismo trovis ian aplikon malgraŭ granda misprezenta formo en Usono, Svisujo k.a. Hodiaŭ ĝi estas moto en la programo de multaj politikaj partioj. La ideo pri unuiĝinta Eŭropo pretendas pri federisma principo. Konsekvence «federistoj» ne mankas hodiaŭ meze de tiu mondo dividita en eterne malpacantaj naciaj ŝtatoj kaj ŝtatetoj.}

\para{Tamen, kiel la aliaj fundamentaj principoj —natura manifestiĝo de la vivo— tiel la federismo starigas kontraŭ si malamikajn fortojn en la nuntempa socia sistemo, kiuj malhelpas al ĝia plena realiĝo. Kiuj kaj kiaj estas tiuj malamikaj fortoj, ni tion vidos en la sekvantaj ĉapitroj.}