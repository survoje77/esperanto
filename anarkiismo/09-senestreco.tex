\section*{Senestreco}
\addcontentsline{toc}{section}{Senestreco}
\indent 
La ĝis tie ĉi pritraktitaj sep principoj estas tiom naturaj, karakterizaj kaj distingaj de la senestreca koncepto en komparo kun ĉiuj aliaj sociaj kaj filozofiaj konceptoj, ke ili kun plena rajto povas esti konsiderataj kiel fundamentaj principoj de anarkiismo. Iliaj historia konstato, eltiro el la vivo kaj scienca argumentado prezentas la fundamentojn de la senestreco. Kaj ĉar la nomo de ĉiu doktrino determiniĝas laŭ ties principoj, t.e. laŭ ties plej alta idealo, ĉiu el tiuj sep principoj povus doni al la doktrino sian nomon. Tiel, kun plena rajto senestreco povus sin nomi «libertalismo» (el la latina nomo de libereco), «solidarismo», «egalrajtismo», «justecismo» (ĉiam laŭ la latina radiko de la vortoj), «interkonsentismo» aŭ «kontraktismo», «iniciativismo» aŭ «aktivismo» kaj precipe «federismo», tiu lasta estus la plej adekvata nomo.

Propre filozofia doktrino povas kontentiĝi nur per konstruo de koncepto sur la bazo de kelkaj principoj. Tio, tamen, ne estas sufiĉa por socia doktrino. Ĉiu socia doktrino, starigante certajn principojn, neeviteble venas en konflikto kun aliaj en la socio jam starigitaj principoj, kaj strebas ilin anstataŭi aŭ modifi, ilin neante entute aŭ parte. Tiel ĉiu socia doktrino strebas aliformiĝi en movadon. Iĝante movado, ĝi neeviteble ekkonfliktas kontraŭ la establitaj sociaj formoj. Same okazis ankaŭ kun anarkiismo.

Anarkiismo ne estas nihilismo, ne estas simpla neado de ĉio ekzistanta, kontraŭe ni pruvis ĝis nun, ke ĉiuj ĝiaj bazaj principoj estas eltiritaj el la nuntempa socio. Sed ilin arigante en strukturita sistemo kaj ilin proklamante sia plej alta celo kaj socia idealo, anarkiismo devis neeviteble refuti serion da aliaj principoj, kiuj same trovas aplikon en la nuntempa socio, kaj tiamaniere ĝi starigis novajn principojn, kiujn oni povas nomi produktaj principoj de la anarkiismo. Unu el tiuj principoj estas senestreco –tiu ĉi, kiu donis nomon al la doktrino kaj al la movado.

La anglo William \qui{Godwin} kiel unua starigis la bazajn principojn de la anarkiismo en 1793, sed li donis neniun nomon al tiu nova doktrino, eltirita aŭ pli precize kaŭzita senpere de la granda franca revolucio. La franco \qui{Prudono} kiel dua, siavice, tute memstare kaj sendepende de GODWIN, elmetis kaj motivis la anarkiismajn principojn en sia unua impona verko Kio estas proprieto? aperinta en 1840, kaj uzis la vorton «anarkio». Li ankaŭ pli malfrue multfoje uzis la saman terminon kiel nomon de la doktrino, kiun li elmetis kaj defendis, sed neniam rekte kaj en determinita formo nomis sin «anarkiisto». Kontraŭe li ĉiam nomis sin kaj parolis kiel socialisto.

La ruso Mikaelo \qui{Bakunin}, elmigrinte en Eŭropon, tria laŭvice elmetas kaj motivas buŝe kaj skribe la principojn de la anarkiismo kaj, kio estas plej grava, transformas kiel unua la doktrinon en movadon. Li same kiel PRUD0N uzas la terminojn «anarkio, anarkiismo, anarkiisto» kaj multe pli ofte ol tiu lasta; sed la doktrinon, kies adepto kaj ĝis alta grado teoriisto li estas, kaj la movadon, kiujn li kreas kun siaj amikoj el la Unua Internacio, li preferas nomi «revolucia socialismo», «kontraŭaŭtoritata socialismo» kaj de tempo al tempo nur «kolektivismo», metante ĝin kontraŭe al la komunismo, kiun ĝis tiam oni konsideris nur kiel regpovan kaj ŝtatan formon de la socia ekonomio.

La nomoj «anarkio, anarkiismo, anarkiisto» estas donitaj pli verdire de la malamikoj ol de la adeptoj de tiu doktrino kaj movado. Longtempe la disciploj kontraŭstariĝis al tiu nomo, ĉiam skribante «an–arkio», kio laŭ greka origino signifas senestreco, foresto de aŭtoritato, kaj ankaŭ senordo. Poste la anarkiistoj malatentis ĉi tiun vorton kaj plu sin nomis mem anarkiistoj, senestrecanoj. La nomo ne estas grava, esenca estas la enhavo.

Ĉiuokaze la termino senestreco kreiĝis el la neado de la aŭtoritato, de la regpovo. Tiel refutante la aŭtoritaton kaj la regpovon, la adeptoj de la sep jam pritraktitaj principoj de la anarkiismo starigas novan principon, eltirita el la neado de la aŭtoritato, ĝuste: senestrecon. Ĉu vere necesa tiu ĉi nova principo por kompletigi la doktrinon al kiu ĝi donis sian nomon? Jes! Ĉar la sep pritraktitaj principoj bazaj, kiuj konsistigas la enhavon de la doktrino mem, estas samtempe la absolutaj kondiĉoj nepre necesaj por la starigo de socio, en kiu la aŭtoritato montriĝas malutila kaj superflua; sed la plena realiĝo eĉ de unu el tiuj bazaj principoj ne estas ebla, sen likvidado, sen neniigo kaj sen forigo de la aŭtoritato de l’ homo super homo, t.e. sen starigo de socio sen aŭtoritato.

Historie la aŭtoritato naskiĝis kaj ideologie motiviĝis kiel rezulto de la ideo pri Dio. La konata frazo enigita en la kristanismon fare de apostolo PAULO, laŭ kiu «ĉia ordonpovo estas de Dio», estas nek hazarda nek tute nova kiel pravigo de la aŭtoritato de l’ homo super la homo. Laŭ ĉiuj religioj, Dio estas absoluta principo. Li estas ĝenerale la plena absoluto: absoluta forto kreanta kaj ĉiokreanta; absoluta vero, plena lumo kaj fonto de ĉia lumo; absoluta saĝeco, senmorteco kaj fonto de ĉia vivo, ktp, ktp… Pro ĉiuj tiuj «absolutoj» Dio–kreinto kaj patro de ĉio kaj de ĉiuj estas ankaŭ absoluta «aŭtoritato». Dubo pri lia vereco kaj senerareco, pri lia justeco kaj racio estas neebla, neakceptebla. Kaj ĉiuj tiuj kvalitoj al la elpensaĵo «Dio» atribuitaj de la homo mem, primitiva kaj nescia (efektiva kreinto de Dio) donas al li nedetrueblan kaj nevenkeblan forton. Ĉiu tia forto estigas timon miksitan kun respekto. «Timo al Dio estas la komenco de ĉia saĝeco» (Ĥristo BOTEV).

Kiel konate el la historio de la religioj, ĉe la originaj formoj de adorado la diaĵo estas prezentita per objektoj kaj fenomenoj videblaj kaj senteblaj. Pli malfrue, kun starigo de diunueco, kiam komenciĝas la vera religio en nuntempa senco de la vorto (kun ĝia socia karaktero), alidirite kiam Dio transformiĝas en nevideblan, ĉieestantan kaj abstraktan estaĵon, aperas ankaŭ la bezono de perantoj, reprezentantoj de Dio. Per tio ideologie formiĝas la ekzistado de la religia kasto. Kiel reprezentanto de Dio ĝi reprezentas parton de ties aŭtoritato. De tie la bezono pri konsekrado de surtera aŭtoritato, kiu trovis esprimon en la konstitucioj de la monarkia ŝtato («reĝo laŭ Di–favoro»). De tio la fama interpreto fare de BOTEV pri la Salomona saĝeco: «Timu Dion, respektu reĝon».

La suprenirado de la burĝaro kiel klaso trudis la neceson de novaj formoj de aŭtoritato –naskis la parlamentismon, kaj kun ĝi la novan religion: universala balotrajto. Sed la ideo pri Dio ne foriĝis, ĝi nur aliformiĝis. La lokon de Dio okupis la nacio, el kiu fariĝis la novnaskita moderna ŝtato. Unu el la fondintoj de la nova religio estis J. J. RUSO per sia elpensita «Socia kontrakto», konforme al kiu la homoj bonvole kaj konscie unuiĝas en ŝtato por defendi sin. Tio ne forigis la bezonon de dieca aŭtoritato. Estiĝis neceso de nur iu kompletigo, kion ni trovas en la formulo: «Reĝo laŭ Di–favoro kaj popola Volo». Sed la fonto ĉiam restis sama: Universala principo, universala saĝeco, aŭtoritato de Dio dividita en malgrandajn parceletojn, el kiuj unu(n) posedas ĉiu voĉdonanto.

Tiel estas la teologia historio de la aŭtoritato. Estas ankaŭ alia historio ne malpli grava, la historio de la perforto, konkerado, milito ktp, pri kiuj ni alifoje okupiĝos. La esenco tamen ĉiam radikiĝas en la dia aŭtoritato. Ĉi–lastan, kvazaŭ de Kreinto postlasitan parteton, ni trovas en la familio reprezentata de la patro. La patra aŭtoritato, laŭ la forto de la sama teologia filozofio, estas sanktaĵo. Kiel konate, ĉe multe da popoloj la patro ĝuis absolutan autoritaton super la infanoj, super ties vivo inkluzive. Li povis ilin mortigi, vendi, kaj senlimige ĉion fari per ili, sen respondeci antaŭ iu ajn. Tiel estas ankaŭ lia morala aŭtoritato absoluta, kiu por multaj restas sendifekta ĝis hodiaŭ.

Aŭtoritato, neniam modifante sian originon el Dio, manifestiĝas per egala forto ankaŭ en la scienco, edukado, arto . La tuta spirita kaj kultura vivo de la nuntempa socio estas venenita de la aŭtoritato, se ne paroli pri la eklezio, tribunalo, armeo, kiujn ni studos aparte.

Tiel, aŭtoritato kun ĝia dieca origino aperas kieI la bazo de ĉiuj formoj de aŭtoritato. Ĉar, kiel la aŭtoritato, tiel ankaŭ la povo en ĉiuj siaj formoj aperas kiel nesupereblaj baroj kontraŭ la plena kaj libera manifestiĝo de la bazaj anarkiismaj principoj, kondiĉoj de ĉia kultura kaj morala progreso, de bonstato kaj feliĉo de l’ homo, la doktrino kaj movado, kiujn proklamas tiuj principoj, kategorie refutas la principon de aŭtoritato kaj ties aplikon en ĉiuj ties formoj, kaj sin proklamas por senaŭtoritata socio –por senestreco,

Anarkio, kiel socia ordo en perfekta kaj plena formo, neniam ekzistis, kaj kiam senestreco manifestiĝis en periodoj de popolaj insurekcioj kaj revolucioj, ĝi ne povis sin teni longdaŭre. Aperinte kiel neado de ĉia aŭtoritato, ĝi plene realigos kaj alprenos daŭran karakteron nur kiam la principo de aŭtoritato kaj aŭtoritataj formoj de sociaj rilatoj estos tute detruitaj de la voloj de la popoloj, kiuj estos dezirintaj realigi la liberecon, solidarecon, egalecon kaj justecon. 